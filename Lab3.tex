%%%%%%%%%%%%%%%%%%%%%%%%%%%%%%%%%%%%%%%%%%%%%%%%%%%%%%%%%%%%%%%%
%                                                              %
% Jared McDaniel                                               %
% ECE 351 Section 51                                           %
% Lab 3                                                        %
% 9/21/2021                                                    %
% This file contains the contents of the Lab 3 Report          %
%                                                              %
%%%%%%%%%%%%%%%%%%%%%%%%%%%%%%%%%%%%%%%%%%%%%%%%%%%%%%%%%%%%%%%%


\documentclass[12pt]{report}
\usepackage[english]{babel}
%\usepackage{natbib}
\usepackage{url}
\usepackage[utf8x]{inputenc}
\usepackage{amsmath}
\usepackage{graphicx}
\graphicspath{{images/}}
\usepackage{parskip}
\usepackage{fancyhdr}
\usepackage{vmargin}
\usepackage{listings}
\usepackage{hyperref}
\usepackage{xcolor}

\definecolor{codegreen}{rgb}{0,0.6,0}
\definecolor{codegray}{rgb}{0.5,0.5,0.5}
\definecolor{codeblue}{rgb}{0,0,0.95}
\definecolor{backcolour}{rgb}{0.95,0.95,0.92}

\lstdefinestyle{mystyle}{
    backgroundcolor=\color{backcolour},   
    commentstyle=\color{codegreen},
    keywordstyle=\color{codeblue},
    numberstyle=\tiny\color{codegray},
    stringstyle=\color{codegreen},
    basicstyle=\ttfamily\footnotesize,
    breakatwhitespace=false,         
    breaklines=true,                 
    captionpos=b,                    
    keepspaces=true,                 
    numbers=left,                    
    numbersep=5pt,                  
    showspaces=false,                
    showstringspaces=false,
    showtabs=false,                  
    tabsize=2
}
 
\lstset{style=mystyle}

\setmarginsrb{3 cm}{2.5 cm}{3 cm}{2.5 cm}{1 cm}{1.5 cm}{1 cm}{1.5 cm}

\title{Lab 3}								
% Title
\author{ Jared McDaniel}						
% Author
\date{9/21/2021}
% Date

\makeatletter
\let\thetitle\@title
\let\theauthor\@author
\let\thedate\@date
\makeatother

\pagestyle{fancy}
\fancyhf{}
\rhead{\theauthor}
\lhead{\thetitle}
\cfoot{\thepage}
%%%%%%%%%%%%%%%%%%%%%%%%%%%%%%%%%%%%%%%%%%%%
\begin{document}

%%%%%%%%%%%%%%%%%%%%%%%%%%%%%%%%%%%%%%%%%%%%%%%%%%%%%%%%%%%%%%%%%%%%%%%%%%%%%%%%%%%%%%%%%

\begin{titlepage}
	\centering
    \vspace*{0.5 cm}
   % \includegraphics[scale = 0.075]{bsulogo.png}\\[1.0 cm]	% University Logo
\begin{center}    \textsc{\Large   ECE 351 - Section 51 }\\[2.0 cm]	\end{center}% University Name
	\textsc{\Large Discrete Convolution  }\\[0.5 cm]				% Course Code
	\rule{\linewidth}{0.2 mm} \\[0.4 cm]
	{ \huge \bfseries \thetitle}\\
	\rule{\linewidth}{0.2 mm} \\[1.5 cm]
	
	\begin{minipage}{0.4\textwidth}
		\begin{flushleft} \large
		%	\emph{Submitted To:}\\
		%	Name\\
          % Affiliation\\
           %contact info\\
			\end{flushleft}
			\end{minipage}~
			\begin{minipage}{0.4\textwidth}
            
			\begin{flushright} \large
			\emph{Submitted By :} \\
			Jared McDaniel  
		\end{flushright}
           
	\end{minipage}\\[2 cm]
	
%	\includegraphics[scale = 0.5]{PICMathLogo.png}
    
    
    
    
	
\end{titlepage}

%%%%%%%%%%%%%%%%%%%%%%%%%%%%%%%%%%%%%%%%%%%%%%%%%%%%%%%%%%%%%%%%%%%%%%%%%%%%%%%%%%%%%%%%%

\tableofcontents
\pagebreak

%%%%%%%%%%%%%%%%%%%%%%%%%%%%%%%%%%%%%%%%%%%%%%%%%%%%%%%%%%%%%%%%%%%%%%%%%%%%%%%%%%%%%%%%%
\renewcommand{\thesection}{\arabic{section}}
\section{Introduction}
 

The goal of this lab is to become more familiar with convolution and its properties using Python. 

\section{Equations}

The equations below portray the ones that were used during this lab. In Part 1 of this lab, these user-defined functions were plotted as a single figure. In Part 2 of this lab, these functions were convolved. We convolved functions f1(t) and f2(t), f2(t) and f3(t), and f1(t) and f3(t). 

\begin{equation}
    f_{1}(t) = u(t−2) − u(t−9)
\end{equation}
\begin{equation}
    f_{2}(t) = e^{-t}u(t)
\end{equation}
\begin{equation}
    f_{3}(t) = r(t-2)[u(t-2) - u(t-3)] + r(4-t)[u(t-3) - u(t-4)]
\end{equation}


\section{Methodology}

This is my code for Part 1 of Lab 3. I started this section of code by copying my defined step and ramp functions from Lab 2. Once that was completed I defined the functions that were given to me in the Lab 3 handout. Then I plotted each function using subplots. The plots for the three user-defined functions can be viewed in Figure 1 under the Results section of this lab report.  
\begin{lstlisting}[language=Python]
#%% Part 1 Code

import numpy as np
import matplotlib.pyplot as plt
import scipy.signal as sig

plt.rcParams.update({ 'font.size' :14}) #font size on plots

steps = 1e-2 #step size
t = np.arange(0, 20 + steps, steps) 
#plot starts at zero and finishes at 10 on the x-axis

def u(t): #step function
    y=np.zeros(t.shape) #y(t) is initialized as an array of zeros
    
    for i in range(len(t)):
        if(t[i]<0):
            y[i]=0 #step function is 0 when t < 0
        else:
            y[i]=1 #step function is 1 when t >= 0
    return y #return outputs stored in the array

def r(t): #ramp function
    y=np.zeros(t.shape) #y(t) is initialized as an array of zeros
    
    for i in range(len(t)):
        if(t[i]<0):
            y[i]=0 #ramp function is 0 when t < 0
        else:
            y[i] = t[i] #ramp function equals t when t >= 0
    return y #return outputs stored in the array

def f1(t): return u(t-2)-u(t-9)
def f2(t): return np.exp(-t)*u(t)
def f3(t): return r(t-2)*(u(t-2)-u(t-3)) + r(4-t)*(u(t-3)-u(t-4))
#can't have brackets, [], in f3(t) equation or the graph won't produce

plt.figure(figsize = (10,7))
#f1(t) plot
plt.subplot(3, 1, 1) #subplot code taken from lab 1 handout. Subplot 1
plt.plot(t, f1(t))
plt.grid()
plt.ylabel('f1(t)')
plt.title('Part 1 Functions')

#f2(t) plot
plt.subplot(3, 1, 2) #subplot code taken from lab 1 handout. Subplot 2
plt.plot(t, f2(t))
plt.grid()
plt.ylabel('f2(t)')

#f3(t) plot
plt.subplot(3, 1, 3) #subplot code taken from lab 1 handout. Subplot 3
plt.plot(t, f3(t))
plt.grid()
plt.ylabel('f3(t)')
plt.xlabel('t') # xlabel only needs to go on the bottom plot
plt.show()
\end{lstlisting}

{This is my code for Part 2 Task 1 of lab 3. For this section of the lab, we were tasked with creating our own code to perform a convolution. My classmates and I were unable to come up with this on our own, so our TA provided the code for us. Without the TA's help, I would not have been able to create this code. The plots that correspond with this code can be viewed in Figure 2 under the Results section of this lab report. I have added plenty on comments in this section of code that should help explain the thought process behind it. }
\begin{lstlisting}[language=Python]
#%% Part 2 Task 1 Code

def c(f1, f2): #defining convolution in python for f1 and f2.
    n1 = len(f1) #returns the number of elements in an array. (horizontal)
                 #returns an integer representing the items in f1 passed as
                 #an argument.
                 #length of array
    n2 = len(f2) #returns the number of elements in an array. (horizontal)
                 #returns an integer representing the items in f1 passed as
                 #an argument.
                 #length of array
    l1 = np.append(f1 , np.zeros((1, n2-1)))
    # np.append adds values to the end of a numpy array.
    # values are appended to a copy of this f1 array.
    # np.zeros defines an array of zeros. In this case, 1 is the shape of the
    # array and n2-1 is the desired data type of the array
    #causes f1 to be the same as f2
    l2 = np.append(f2 , np.zeros((1, n1-1)))
    # np.append adds values to the end of a numpy array.
    # values are appended to a copy of this f2 array.
    # np.zeros defines an array of zeros. In this case, 1 is the shape of the
    # array and n2-1 is the desired data type of the array
    #causes f2 to be the same as f1
    result = np.zeros(l1.shape) #creates an array of zeros with the l1 shape
    
    for i in range(n2+n1-2): #range creates a range of numbers that is nice
                             #for for loops
                             #adds n2 and n1((n2-1) + (n1-1))
                             #i goes through both
        result[i] = 0 #initializes all of them to be zero
        for j in range(n1): #when j is in the range of n1, if i-j+1 is > 0
                            #result[i] will ne added to the right side of the 
                            #equation.
            if (i - j + 1 > 0): 
            #if length of both functions is greater than the first function
                try:
                    result[i] += l1[j]*l2[i-j+1] 
                    #adds duration of each function
                except:
                    print(i,j)
    return result*steps #returns result value



N = len(t)  #returns the number of elements in an array. (horizontal)
            #returns an integer representing the items in t passed as an
            #argument.
t1 = np.arange(0, 2*t[N-1], steps)
#create a numpy array that is a range. My interval starts at
# 0 and goes to 2*t[N-1]. The steps are values at 1e-2.    
f1=f1(t) #setting f1 = to my f1(t) function
f2=f2(t) #setting f2 = to my f2(t) function
f3=f3(t) #setting f3 = to my f3(t) function

plt.figure(figsize = (10,7))
#f1(t) plot
plt.subplot(3, 1, 1) #subplot code taken from lab 1 handout. Subplot 1
plt.plot(t1, c(f1,f2))
plt.ylabel('Convolve f1(t) and f2(t)')
plt.title('Part 2 Task 1 Convolutions')
plt.grid()

#f2(t) plot
plt.subplot(3, 1, 2) #subplot code taken from lab 1 handout. Subplot 2
plt.plot(t1, c(f2,f3))
plt.grid()
plt.ylabel('Convolve f2(t) and f3(t)')

#f3(t) plot
plt.subplot(3, 1, 3) #subplot code taken from lab 1 handout. Subplot 3
plt.plot(t1, c(f1,f3))
plt.grid()
plt.ylabel('Convolve f1(t) and f3(t)')
plt.xlabel('t') # xlabel only needs to go on the bottom plot
plt.show()
\end{lstlisting}

{This is my code for Part 2 Task 2, 3, and 4 of lab 3. For this section of the lab, we were tasked with using the \textbf{sig.convolve()} function to verify our Part 2 Task 1 code. The code is quite repetitive, but very easy to follow. We convolved the user defined functions we created in Part 1 of this lab. All the plots that I obtained from this code were identical to the plots in the first task of Part 2. Figures 3, 4, and 5 portray the plots that correspond with this code.}
\begin{lstlisting}[language=Python]
#%% Part 2 Task 2, 3, and 4 Code

c1 = sig.convolve(f1, f2)*steps #convolution of functions 1 and 2

plt.figure(figsize = (10,7))
plt.plot(t1, c1)
plt.grid()
plt.title ('Part 2 Task 2 Plot')
plt.ylabel('Convolution of f1(t) and f2(t)')
plt.xlabel('t')
plt.show()

c2 = sig.convolve(f2, f3)*steps #convolution of functions 2 and 3

plt.figure(figsize = (10,7))
plt.plot(t1, c2)
plt.grid()
plt.title ('Part 2 Task 3 Plot')
plt.ylabel('Convolution of f2(t) and f3(t)')
plt.xlabel('t')
plt.show()

c3 = sig.convolve(f1, f3)*steps #convolution of functions 1 and 3

plt.figure(figsize = (10,7))
plt.plot(t1, c3)
plt.grid()
plt.title ('Part 2 Task 4 Plot')
plt.ylabel('Convolution of f1(t) and f3(t)')
plt.xlabel('t')
plt.show()
\end{lstlisting}



\section{Results}

The results for this lab are shown below as plots. The first figure shows the plots from the user-defined functions in Part 1 of the lab. The second figure portrays three plots of the user-defined convolutions. The first plot is the convolution of f1(t) and f2(t), the second plot is the convolution of f2(t) and f3(t), and the third plot is the convolution of f1(t) and f3(t). The last three figures are identical to the user-defined convolution plots, but they were created using a \textbf{sig.convolve()} function instead of the user-defined convolution code. I have given each plot a title, a x-axis title, and a y-axis title. The lab did work as I expected it to. I did not get any odd graphs from my code, and my TA confirmed that. 



\begin{figure}
\includegraphics[width=\linewidth]{Part1Functions.png}
\caption{Part 1 User-Defined Functions}
\end{figure}

\begin{figure}
\includegraphics[width=\linewidth]{Part2Task1.png}
\caption{User-Defined Convolution}
\end{figure}

\begin{figure}
\includegraphics[width=\linewidth]{Part2Task2.png}
\caption{Convolution of F1 and F2}
\end{figure}

\begin{figure}
\includegraphics[width=\linewidth]{Part2Task3.png}
\caption{Convolution of F2 and F3}
\end{figure}

\begin{figure}
\includegraphics[width=\linewidth]{Part2Task4.png}
\caption{Convolution of F1 and F3}
\end{figure}


\newpage

\section{Error Analysis}

When working through this lab, I did struggle with creating our own code to perform a convolution. I was unsure how to start and finish the code. If our TA wasn't there to help us through it, I would have not completed this part of the lab. However, I do think collaborating with my classmates was beneficial to me as a student. We were able to come up with an idea together that logically made sense. 



\section{Questions}

\textbf{1. Did you work alone or with classmates on this lab? If you collaborated to get to the solution,
what did that process look like?}
{I worked with my classmates for Task 1 of Part 2 of this lab. I completed the other parts/tasks from this lab on my own. When collaborating to obtain the solution, we used one of the whiteboards in the lab and began brainstorming on what we should do. I believe we were able to create a solid plan, but we were unsure how to implement it into our code. After thirty minutes of brainstorming, the TA was able to help us obtain the correct solution.}

\textbf{2. What was the most difficult part of this lab for you, and what did your problem-solving
process look like?}
{The most difficult part of this lab was creating our own code to perform a convolution. My problem-solving process began with asking my roommate for help. He has already taken this lab, and I thought he could get me going in the right direction. However, that was not the case. He was able to give me some ideas, but this part of the lab was a struggle for him too. So, I decided I would collaborate with my classmates and try to find the answer. Unfortunately, that also failed and the TA helped us obtain the solution. }

\textbf{3. Did you approach writing the code with analytical or graphical convolution in mind? Why
did you chose this approach?}
{I approached writing this code with a graphical convolution in mind. This approach allowed for a visual aid, which is extremely helpful for me when it comes to writing code.}

\textbf{4. Leave any feedback on the clarity of the expectations, instructions, and deliverables.}
{I think everything was clarified well in the lab 3 document. I am unsure why we had to write our own code to perform a convolution, but the process was interesting to think through. I am certain I would have never been able to complete this lab if our TA didn't help us through it. Other than that, the lab was very clear and easy to complete. }

\newpage

\section{Conclusion}

During this lab, I learned how to convolve functions using python. I do feel that the lab was successful, however I still do not understand why we had to create our own code for a convolution. I know I wouldn't have been able to obtain the correct code for that section of the lab if the TA didn't help us, and I still don't fully understand how that code works. Though that may be the case, I still think the lab was helpful in furthering my knowledge about convolutions and python. If I didn't brainstorm with my classmates, convolution would still be a struggle for me. Here is the link to my github: \href{https://github.com/JMac1999}{Click Here}

\newpage


\begin{thebibliography}{111}

  \bibitem{ACMT}
A. Aldroubi, C. Cabrelli, U. Molter, and Sui Tang,
Dynamical sampling, 
{\it  Applied and Computational Harmonic Analysis}, doi:10.1016/j.acha.2015.08.014, 2016


\end{thebibliography}
\end{document}

