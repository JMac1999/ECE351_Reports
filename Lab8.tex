%%%%%%%%%%%%%%%%%%%%%%%%%%%%%%%%%%%%%%%%%%%%%%%%%%%%%%%%%%%%%%%%
%                                                              %
% Jared McDaniel                                               %
% ECE 351 Section 51                                           %
% Lab 8                                                        %
% 10/26/2021                                                   %
% This file contains the contents of the Lab 8 Report          %
%                                                              %
%%%%%%%%%%%%%%%%%%%%%%%%%%%%%%%%%%%%%%%%%%%%%%%%%%%%%%%%%%%%%%%%


\documentclass[12pt]{report}
\usepackage[english]{babel}
%\usepackage{natbib}
\usepackage{url}
\usepackage[utf8x]{inputenc}
\usepackage{amsmath}
\usepackage{graphicx}
\graphicspath{{images/}}
\usepackage{parskip}
\usepackage{fancyhdr}
\usepackage{vmargin}
\usepackage{listings}
\usepackage{hyperref}
\usepackage{xcolor}

\definecolor{codegreen}{rgb}{0,0.6,0}
\definecolor{codegray}{rgb}{0.5,0.5,0.5}
\definecolor{codeblue}{rgb}{0,0,0.95}
\definecolor{backcolour}{rgb}{0.95,0.95,0.92}

\lstdefinestyle{mystyle}{
    backgroundcolor=\color{backcolour},   
    commentstyle=\color{codegreen},
    keywordstyle=\color{codeblue},
    numberstyle=\tiny\color{codegray},
    stringstyle=\color{codegreen},
    basicstyle=\ttfamily\footnotesize,
    breakatwhitespace=false,         
    breaklines=true,                 
    captionpos=b,                    
    keepspaces=true,                 
    numbers=left,                    
    numbersep=5pt,                  
    showspaces=false,                
    showstringspaces=false,
    showtabs=false,                  
    tabsize=2
}
 
\lstset{style=mystyle}

\setmarginsrb{3 cm}{2.5 cm}{3 cm}{2.5 cm}{1 cm}{1.5 cm}{1 cm}{1.5 cm}

\title{Lab 8}								
% Title
\author{ Jared McDaniel}						
% Author
\date{10/26/2021}
% Date

\makeatletter
\let\thetitle\@title
\let\theauthor\@author
\let\thedate\@date
\makeatother

\pagestyle{fancy}
\fancyhf{}
\rhead{\theauthor}
\lhead{\thetitle}
\cfoot{\thepage}
%%%%%%%%%%%%%%%%%%%%%%%%%%%%%%%%%%%%%%%%%%%%
\begin{document}

%%%%%%%%%%%%%%%%%%%%%%%%%%%%%%%%%%%%%%%%%%%%%%%%%%%%%%%%%%%%%%%%%%%%%%%%%%%%%%%%%%%%%%%%%

\begin{titlepage}
	\centering
    \vspace*{0.5 cm}
   % \includegraphics[scale = 0.075]{bsulogo.png}\\[1.0 cm]	% University Logo
\begin{center}    \textsc{\Large   ECE 351 - Section 51 }\\[2.0 cm]	\end{center}% University Name
	\textsc{\Large Fourier Series Approximation of a Square Wave  }\\[0.5 cm]				% Course Code
	\rule{\linewidth}{0.2 mm} \\[0.4 cm]
	{ \huge \bfseries \thetitle}\\
	\rule{\linewidth}{0.2 mm} \\[1.5 cm]
	
	\begin{minipage}{0.4\textwidth}
		\begin{flushleft} \large
		%	\emph{Submitted To:}\\
		%	Name\\
          % Affiliation\\
           %contact info\\
			\end{flushleft}
			\end{minipage}~
			\begin{minipage}{0.4\textwidth}
            
			\begin{flushright} \large
			\emph{Submitted By :} \\
			Jared McDaniel  
		\end{flushright}
           
	\end{minipage}\\[2 cm]
	
%	\includegraphics[scale = 0.5]{PICMathLogo.png}
    
    
    
    
	
\end{titlepage}

%%%%%%%%%%%%%%%%%%%%%%%%%%%%%%%%%%%%%%%%%%%%%%%%%%%%%%%%%%%%%%%%%%%%%%%%%%%%%%%%%%%%%%%%%

\tableofcontents
\pagebreak

%%%%%%%%%%%%%%%%%%%%%%%%%%%%%%%%%%%%%%%%%%%%%%%%%%%%%%%%%%%%%%%%%%%%%%%%%%%%%%%%%%%%%%%%%
\renewcommand{\thesection}{\arabic{section}}
\section{Introduction}
 

The goal of this lab is to become familiar with the usage of Fourier Series to approximate periodic time-domain signals.


   

\section{Fourier Series Equations}
{The equations below portray the ones that were given to us in the lab handout. Equation 1 represents a real-value function represented by a Fourier series. Equations 2 and 3 represent the values of ak and bk.} 
 
 \begin{equation}
    x(t) = \frac{1}{2}a_{0}+\sum_{k=1}^n a_{k}\cos{(k\omega_{0}t)}+b_{k}\sin{(k\omega_{0}t)}
\end{equation}
\begin{equation}
    a_{k} = \frac{2}{T}\int_{0}^T x(t)\cos{(k\omega_{0}t)}dt
\end{equation}
\begin{equation}
    b_{k} = \frac{2}{T}\int_{0}^T x(t)\sin{(k\omega_{0}t)}dt
\end{equation}
\begin{equation}
   \omega_{0} = \frac{2\pi}{T}
\end{equation}


\section{Expressions for ak, bk, and x(t)}
{Since ak is a cosine function, it will equal zero. The function is also asymmetrical. Equation 6 represents the value of bk. We printed the values of bk in this lab, and those can seen in Figure 1 under the Results section of this lab report. Equation 7 and 8 represent the real-value function with the calculated values of a0, ak, and bk. Figures 2 and 3 under the results section of this lab report represent the plots of Equation 8 for six different values of N. The N values we used were 1, 3, 15, 50, 150, 1500.}
\begin{equation}
    b_{k} = \frac{2}{T}\int_{0}^T x(t)\sin{(\frac{kt2\pi}{T})}dt
\end{equation}
{At T = 1, x(t) = 1.}
\begin{equation}
    b_{k} = \frac{1}{\pi k}(1 - \cos{2\pi k})
\end{equation}
\begin{equation}
   x(t) = 0 + \sum_{k=1}^\infty 0 + b_{k}\sin{(k\omega_{0}t)}
\end{equation}
\begin{equation}
   x(t) = \sum_{k=1}^\infty \frac{1}{\pi k}(1 - \cos{2\pi k}) \cdot \sin{(k\omega_{0}t)}
\end{equation}



\section{Methodology}

This is my code for Part 1 of Lab 8. I started this section of code by defining my bk and b(t, k) equations. I used equations 6 and 8 from this lab report for my equation definitions. Then I used the print command to print the first three values of bk, and the first two values of ak. Those printed values can be seen in Figure 1 under the Results section of this lab report.
\begin{lstlisting}[language=Python]
import numpy as np
import matplotlib.pyplot as plt
import scipy.signal as sig
from scipy.integrate import quad

plt.rcParams.update({ 'font.size' :14}) #font size on plots

steps = 1.2e-3 #step size
t = np.arange(0, 20 + steps, steps) 
#plot starts at zero and finishes at 20 on the x-axis

#Part 1 Task 1 Code

def b(t, k):
    return (2/(k*3.14153))*(1-np.cos(k*3.14153))*np.sin(k*(2*3.14153/8)*t)
# T = 8s
def b_k(k):
    return (2/(k*3.14153))*(1-np.cos(k*3.14153))
    
print('a_0 = ', 0)
print('a_1 = ', 0)
print('b_1 = ', b_k(1))
print('b_2 = ', b_k(2))
print('b_3 = ', b_k(3))

\end{lstlisting}

{This is my code for Part 1 Task 2 of Lab 8. For this section of code we were tasked with defining the Fourier series of the square wave given to us in the lab handout. I defined this by incrementing my b(t, k) equation in a for loop. Then I used the subplot code from lab 2 to create my plots for the six given N values. Figure 2 under the Results section of this lab report contains the first three N values, and Figure 3 contains the last 3 N values.  }
\begin{lstlisting}[language=Python]
#Part 1 Task 2 Code and plots

def FS(t, N):
    y = 0
    for k in range(1,N+1):
        y += b(t, k)
    return y

plt.figure(figsize = (10,7))
#N = 1 plot
plt.subplot(3, 1, 1) #subplot code taken from lab 1 handout. Subplot 1
plt.plot(t, FS(t,1))
plt.grid()
plt.ylabel('N = 1')
plt.title('Part 1 Task 2 Plotting Functions')

#N = 3 plot
plt.subplot(3, 1, 2) #subplot code taken from lab 1 handout. Subplot 2
plt.plot(t, FS(t,3))
plt.grid()
plt.ylabel('N = 3')

#N = 15 plot
plt.subplot(3, 1, 3) #subplot code taken from lab 1 handout. Subplot 3
plt.plot(t, FS(t,15))
plt.grid()
plt.ylabel('N = 15')
plt.xlabel('t') # xlabel only needs to go on the bottom plot
plt.show()


plt.figure(figsize = (10,7))
#N = 50 plot
plt.subplot(3, 1, 1) #subplot code taken from lab 1 handout. Subplot 1
plt.plot(t, FS(t,50))
plt.grid()
plt.ylabel('N = 50')
plt.title('Part 1 Task 2 Plotting Functions')

#N = 150 plot
plt.subplot(3, 1, 2) #subplot code taken from lab 1 handout. Subplot 2
plt.plot(t, FS(t,150))
plt.grid()
plt.ylabel('N = 150')

#N = 1500 plot
plt.subplot(3, 1, 3) #subplot code taken from lab 1 handout. Subplot 3
plt.plot(t, FS(t,1500))
plt.grid()
plt.ylabel('N = 1500')
plt.xlabel('t') # xlabel only needs to go on the bottom plot
plt.show()  
\end{lstlisting}

\newpage

\section{Results}

The results for this lab are shown below as printed values and plots. Figure 1 portrays the printed values of a0, a1, b1, b2, and b3. Figure 2 portrays the plots for N = 1, 3, and 15. Figure 3 portrays the plots for N = 50, 150, and 1500. 


\begin{figure}

\includegraphics[width=\linewidth]{Part1Task1PrintedValues.PNG}
\caption{Printed Values for ak and bk}
\end{figure}

\begin{figure}
\includegraphics[width=\linewidth]{Part1Task2Plots.png}
\caption{Fourier Series Plot (N = 1, 3, 15) }
\end{figure}

\begin{figure}
\includegraphics[width=\linewidth]{Part1Task2Plots2.png}
\caption{Fourier Series Plot (N = 50, 150, 1500)}
\end{figure}

\newpage

\section{Error Analysis}

When working through this lab, I did experience difficulty with my final plot where N = 1500. For some reason the plot would either be above 0 or below it. To fix this I had to input an "approximation" of pi into the equations in my code. I'm not sure why this was happening, but the approximation I inputted was 3.14153. Other than that one issue, I thought the lab provided clear explanations that helped me complete it without any issues. 



\section{Questions}

\textbf{1. Is x(t) an even or an odd function? Explain why.}
{x(t) is clearly an odd function. This is true because, the square wave replicates a sin function and it is asymmetrical. }

\textbf{2. Based on your results from Task 1, what do you expect the values of a2, a3, . . . , an to be? Why?}
{Since the function we are plotting is odd, all a values will equal zero. }

\textbf{3.  How does the approximation of the square wave change as the value of N increases? In what way does the Fourier series struggle to approximate the square wave?}
{The approximation of the square wave becomes more accurate as we increase our N values. Since the Fourier series is a sin function, it struggles to approximate a square wave during the rising a falling edges. This is shown in the final three plots from this lab. }

\textbf{4. What is occurring mathematically in the Fourier series summation as the value of N increases?}
{As the value of N increases, the Fourier Series has to computes the sum of more sin waves. This is all an attempt to make a perfect square wave. }

\textbf{5. Leave any feedback on the clarity of the expectations, instructions, and deliverables.}
{I think everything was clarified well in the Lab 8 document. The only task I struggled with in this lab was obtaining a correct plot for N = 1500. Other than that, I thought the lab was easy to follow and complete.}



\section{Conclusion}

During this lab, I became more familiar with Fourier Series, and I became better at using them to approximate periodic time-domain signals. I felt that the lab was successful in its goal of furthering our understanding of using Fourier Series to approximate a square wave. I don't have any recommendations to change this lab. I thought the lab provided all the necessary information for us to complete it. Here is the link to my github: \href{https://github.com/JMac1999}{Click Here}

\newpage


\begin{thebibliography}{111}

  \bibitem{ACMT}
A. Aldroubi, C. Cabrelli, U. Molter, and Sui Tang,
Dynamical sampling, 
{\it  Applied and Computational Harmonic Analysis}, doi:10.1016/j.acha.2015.08.014, 2016


\end{thebibliography}
\end{document}

