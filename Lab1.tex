\documentclass[11pt]{article}
\usepackage{fullpage}

%%%%%%%%%%%%%%%%%%%%%%%%%%%%%%%%%%%%%%%%%%%%%%%%%%%%%%%%%%%%%%%%
%                                                              %
% Jared McDaniel                                               %
% ECE 351 Section 51                                           %
% Lab 1                                                        %
% 9/7/2021                                                     %
% This file contains the questions, answer, and summaries      %
% from Lab 1                                                   %
%%%%%%%%%%%%%%%%%%%%%%%%%%%%%%%%%%%%%%%%%%%%%%%%%%%%%%%%%%%%%%%%


\title{ECE 351 - Section 51 - Lab 1 Questions and Summaries}
\author{Jared McDaniel}
\date{September 7, 2021}


\begin{document}
\maketitle


\section{Part 1 Summary}
\textbf{In this section of lab 1, we were tasked with becoming more familiar with Spyder and Python. To complete this task we had to read information at https://docs.spyder-ide.org/overview.html. Then we had to read over the Spyder shortcut cheat sheet. Our TA sent us the webiste link and the cheat sheet because BBLearn is not working properly yet. After this was done we had open a new Spyder file and name it using the following format:  LastName, FirstName, ECE351,Lab1. The commas indicate underscores.
}

\section{Part 2 Summary}
\textbf{In this section of lab 1, we were tasked with becoming more familiar with Python and Spyder. However, Part 2 of lab 1 was much more detailed than Part 1. In it, we were introduced to methods, and useful operations for defining variables, arrays, and matrices. We were also introduced to Python syntax, packages, and proper plot formatting. Finally, we were introduced to syntax for complex numbers using 'numpy.', and simple commands in Python. This part of the lab provided numerous amounts of example code that we could use in order to familiarize ourselves with Python and Spyder.}

\section{Part 3 Summary}
\textbf{In this section of lab 1, we were introduced to pep8 coding practices and were tasked with becoming more familiar with them. This part of the lab provided example code for proper indentations, defining particular functions or programs, wrapping lines exceeding 79 characters, inserting comments, using spaces around operators and commas, and proper naming conventions. A link for more information on these topics was added at the end of Part 3. The link is as follows:  https://pep8.org/.}

\section{Part 4 Summary}
\textbf{For the final part of lab 1, we were tasked with becoming more familiar with LATEX commands. Part 4 of lab 1 provided LATEX code for package imports. This can be used to create a document to build our own lab reports. Though this was provided, it is also stated that we can find a template that we like to create our reports.}

\section{Lab 1 Questions}
\subsection{1. Which course are you most excited for in your degree? Which course have you enjoyed the most so far?}
\textbf{The courses I am most excited for are the power courses given by this university. As of right now, I am in ECE 404, 420, and 421. The courses I have enjoyed the most so far are ECE 320, and ENGR 350.}
\subsection{2. Leave any feedback on the clarity of the expectations, instructions, and deliverables.}
\textbf{I think everything was clarified well in the lab 1 and lab 0 documents. I did not struggle completing them. I may need more help on Latex when we have to start creating formal reports, but everything else was clear.}

\end{document}