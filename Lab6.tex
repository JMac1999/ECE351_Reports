%%%%%%%%%%%%%%%%%%%%%%%%%%%%%%%%%%%%%%%%%%%%%%%%%%%%%%%%%%%%%%%%
%                                                              %
% Jared McDaniel                                               %
% ECE 351 Section 51                                           %
% Lab 6                                                        %
% 10/12/2021                                                   %
% This file contains the contents of the Lab 6 Report          %
%                                                              %
%%%%%%%%%%%%%%%%%%%%%%%%%%%%%%%%%%%%%%%%%%%%%%%%%%%%%%%%%%%%%%%%


\documentclass[12pt]{report}
\usepackage[english]{babel}
%\usepackage{natbib}
\usepackage{url}
\usepackage[utf8x]{inputenc}
\usepackage{amsmath}
\usepackage{graphicx}
\graphicspath{{images/}}
\usepackage{parskip}
\usepackage{fancyhdr}
\usepackage{vmargin}
\usepackage{listings}
\usepackage{hyperref}
\usepackage{xcolor}

\definecolor{codegreen}{rgb}{0,0.6,0}
\definecolor{codegray}{rgb}{0.5,0.5,0.5}
\definecolor{codeblue}{rgb}{0,0,0.95}
\definecolor{backcolour}{rgb}{0.95,0.95,0.92}

\lstdefinestyle{mystyle}{
    backgroundcolor=\color{backcolour},   
    commentstyle=\color{codegreen},
    keywordstyle=\color{codeblue},
    numberstyle=\tiny\color{codegray},
    stringstyle=\color{codegreen},
    basicstyle=\ttfamily\footnotesize,
    breakatwhitespace=false,         
    breaklines=true,                 
    captionpos=b,                    
    keepspaces=true,                 
    numbers=left,                    
    numbersep=5pt,                  
    showspaces=false,                
    showstringspaces=false,
    showtabs=false,                  
    tabsize=2
}
 
\lstset{style=mystyle}

\setmarginsrb{3 cm}{2.5 cm}{3 cm}{2.5 cm}{1 cm}{1.5 cm}{1 cm}{1.5 cm}

\title{Lab 6}								
% Title
\author{ Jared McDaniel}						
% Author
\date{10/12/2021}
% Date

\makeatletter
\let\thetitle\@title
\let\theauthor\@author
\let\thedate\@date
\makeatother

\pagestyle{fancy}
\fancyhf{}
\rhead{\theauthor}
\lhead{\thetitle}
\cfoot{\thepage}
%%%%%%%%%%%%%%%%%%%%%%%%%%%%%%%%%%%%%%%%%%%%
\begin{document}

%%%%%%%%%%%%%%%%%%%%%%%%%%%%%%%%%%%%%%%%%%%%%%%%%%%%%%%%%%%%%%%%%%%%%%%%%%%%%%%%%%%%%%%%%

\begin{titlepage}
	\centering
    \vspace*{0.5 cm}
   % \includegraphics[scale = 0.075]{bsulogo.png}\\[1.0 cm]	% University Logo
\begin{center}    \textsc{\Large   ECE 351 - Section 51 }\\[2.0 cm]	\end{center}% University Name
	\textsc{\Large Partial Fraction Expansion  }\\[0.5 cm]				% Course Code
	\rule{\linewidth}{0.2 mm} \\[0.4 cm]
	{ \huge \bfseries \thetitle}\\
	\rule{\linewidth}{0.2 mm} \\[1.5 cm]
	
	\begin{minipage}{0.4\textwidth}
		\begin{flushleft} \large
		%	\emph{Submitted To:}\\
		%	Name\\
          % Affiliation\\
           %contact info\\
			\end{flushleft}
			\end{minipage}~
			\begin{minipage}{0.4\textwidth}
            
			\begin{flushright} \large
			\emph{Submitted By :} \\
			Jared McDaniel  
		\end{flushright}
           
	\end{minipage}\\[2 cm]
	
%	\includegraphics[scale = 0.5]{PICMathLogo.png}
    
    
    
    
	
\end{titlepage}

%%%%%%%%%%%%%%%%%%%%%%%%%%%%%%%%%%%%%%%%%%%%%%%%%%%%%%%%%%%%%%%%%%%%%%%%%%%%%%%%%%%%%%%%%

\tableofcontents
\pagebreak

%%%%%%%%%%%%%%%%%%%%%%%%%%%%%%%%%%%%%%%%%%%%%%%%%%%%%%%%%%%%%%%%%%%%%%%%%%%%%%%%%%%%%%%%%
\renewcommand{\thesection}{\arabic{section}}
\section{Introduction}
 

The goal of this lab is to use scipy.signal.residue() to perform partial fraction expansion.

\section{Equations}

The equations below portray the ones that were used during this lab. Equation 1 represents the step response that we found in the prelab. The plot for equation 1 can be seen in Figure 1 under the Results section of this lab report. Equation 2 represents the transfer function we found in the prelab. The plot for this equation can be seen in Figure 2 under the Results section of this lab report. Equation 3 represents our partial fraction expansion equation. The outputs can be seen in Figure 3. Equation 4 represents the differential equation given to us in the lab handout. The outputs can be seen in Figure 4. Equation 5 is the defined Cosine method we learned in class. This equation corresponds with Figure 5 in the Results section of this lab report. Equations 6 and 7 are two equations that I used to help define my Cosine method.  

\begin{equation}
    y(t) = (\frac{1}{2}-\frac{1}{2} e^{-4t}+e^{-6t})u(t)
\end{equation}
\begin{equation}
    H(s) = 1+\frac{2}{s+4}-\frac{6}{s+6}
\end{equation}
\begin{equation}
   Y(s) = \frac{1}{2*s}-\frac{1}{2*(s+4)}-\frac{1}{s}+\frac{1}{s+6}
\end{equation}
\begin{equation}
   y^{(5)}(t) + 18y^{(4)}(t) + 218y^{(3)}(t) + 2036y^{(2)}(t) + 9085y^{(1)}(t) + 25250y(t) = 25250x(t)
\end{equation}
\begin{equation}
   Cosine Method = 2|k|\cdot e^{\alpha t} \cdot \cosin({\omega t + kangle})
\end{equation}
\begin{equation}
    p=\alpha + j\omega
\end{equation}
\begin{equation}
    k=\frac{F_{0}(s)}{(s-p^{*})} = |k|\cdot kangle
\end{equation}


\section{Methodology}

This is my code for Part 1 of Lab 6. I started this section of code by copying my defined step function from Lab 4. Then I defined my hand calculated step response (y(t)) from the prelab. Then I plotted the function in python. The plot for the hand-solved time-domain impulse response can be viewed in Figure 1 under the Results section of this lab report.  
\begin{lstlisting}[language=Python]
#Part 1 Task 1 Code

import numpy as np
import matplotlib.pyplot as plt
import scipy.signal as sig

plt.rcParams.update({ 'font.size' :14}) #font size on plots

steps = 1.2e-5 #step size
t = np.arange(0, 2 + steps, steps) 
#plot starts at zero and finishes at 2 on the x-axis

def u(t): #step function
    y=np.zeros(t.shape) #y(t) is initialized as an array of zeros
    
    for i in range(len(t)):
        if(t[i]<0):
            y[i]=0 #step function is 0 when t < 0
        else:
            y[i]=1 #step function is 1 when t >= 0
    return y #return outputs stored in the array

def y(t): #hand calculated step resonse

    return (1/2 - 1/2*np.exp(-4*t) + np.exp(-6*t))*u(t)

plt.figure(figsize = (10,7))
#Hand Calculated Step Response plot
plt.plot(t, y(t))
plt.grid()
plt.ylabel('y(t)')
plt.xlabel('t')
plt.title('Part 1 Hand Calculated Step Response')

\end{lstlisting}

{This is my code for Part 1 Task 2 of Lab 6. For this section of the lab, we were tasked with using the sig.step function to plot the step response. To do this, I first copied the matrix code from Lab 5 and inputed the values for my numerator and denominator. Then I copied the code I needed to plot a function from part 1 of this lab. The plot for this section of the lab can be viewed in Figure 2 under the Results section of this lab report. This figure will look exactly like Figure 1 from the Results section.  }
\begin{lstlisting}[language=Python]
#Part 1 Task 2 Code

num = [1, 6 , 12] # Creates a matrix for the numerator
den = [1 , 10 , 24] # Creates a matrix for the denominator

tout, yout = sig.step((num,den), T = t)

plt.figure(figsize = (10,7))
#Hand Calculated Step Response plot
plt.plot(tout, yout)
plt.grid()
plt.ylabel('yout')
plt.xlabel('tout')
plt.title('Part 1 Task 2 Sig.Step Response')
\end{lstlisting}



{This is my code for Part 1 Task 3 of Lab 6. In this part of the lab we were tasked with printing our partial fraction expansion results, R, P, and K, using sig.residue. To do this, I set the values for my numerator and denominator. Then I set R, P, and K equal to my sig.residue function. Then I used the print command to print the values of R and P. The values can be viewed in Figure 3 under the Results section of this lab report. }
\begin{lstlisting}[language=Python]
#Part 1 Task 3 Code

num = [1, 6 , 12] # Creates a matrix for the numerator
den = [1 , 10 , 24, 0] # Creates a matrix for the denominator

R, P, K = sig.residue(num, den)

print('R1 = ', R, '\nP1 = ', P)
\end{lstlisting}


{This is my code for Part 2 Task 1 of Lab 6. In this part of the lab we were tasked with printing our partial fraction expansion results, R, P, and K, for the equation given to us in the lab handout. To do this, I first updated my time domain. Then I set the values for my numerator and denominator. Then I set R, P, and K equal to my sig.residue function. Then I used the print command to print the values of R and P. The values can be viewed in Figure 4 under the Results section of this lab report. }
\begin{lstlisting}[language=Python]
#Part 2 Task 1 Code

steps = 1.2e-5 #step size
t = np.arange(0, 4.5 + steps, steps) 
#plot starts at zero and finishes at 2 on the x-axis

num = [25250] # Creates a matrix for the numerator
den = [1 , 18 , 218, 2036, 9085, 25250, 0] # Creates a matrix for the denominator

R, P, K = sig.residue(num, den)

print('R2 = ', R, '\nP2 = ', P)
\end{lstlisting}


{This is my code for Part 2 Task 2 of Lab 6. In this part of the lab we were tasked with plotting the time domain response of the equation given to us in the lab handout using the Cosine method. To do this, I used my class textbook to find all the necessary equations and values I would need for the Cosine method. Then I plotted the function. The plot can be viewed in Figure 5 under the Results section of this lab report. }
\begin{lstlisting}[language=Python]
#Part 2 Task 2 Code

def Cosine(R, P, t): #defined cosine method from the class text
    y = 0;
    for i in range(len(R)):
        mag_k = np.abs(R[i]) #magnitude of k
        phase = np.angle(R[i]) #phase angle of k
        alpha = np.real(P[i]) #alpha is the real component of P
        omega = np.imag(P[i]) #omega is the imaginary component of P
        y += mag_k*np.exp(alpha*t)*np.cos(omega*t + phase)*u(t)
    return y

plt.figure(figsize = (10,7))
#Time Domain Response plot
plt.plot(t, Cosine(R, P, t))
plt.grid()
plt.ylabel('y')
plt.xlabel('t')
plt.title('Part 2 Task 2 Time Domain Response')
\end{lstlisting}

\newpage

{This is my code for Part 2 Task 3 of Lab 6. In this part of the lab we were tasked with checking the H(s) function we used in Part 2 Task 1 of Lab 6 using sig.step. To do this, I copied the code from the Part 2 Task 1 of this lab and used sig.step instead of sig.residue. Then I plotted the function. The plot can be viewed in Figure 6 under the Results section of this lab report. }
\begin{lstlisting}[language=Python]
#Part 2 Task 3 Code

num = [25250] # Creates a matrix for the numerator
den = [1 , 18 , 218, 2036, 9085, 25250] # Creates a matrix for the denominator

tout, yout = sig.step((num,den), T = t)
        
plt.figure(figsize = (10,7))
#Time Domain Response plot
plt.plot(tout, yout)
plt.grid()
plt.ylabel('yout')
plt.xlabel('tout')
plt.title('Part 2 Task 3 Sig.Step Response')   
\end{lstlisting}

\section{Results}

The results for this lab are shown below as plots. Figure 1 portrays the plot of the hand-solved step response function from the prelab. Figure 2 portrays the plot of the step response when using the sig.step function. Figure 3 portrays the printed partial fraction expansion results R, P, and K using the sig.residue function. Figure 4 portrays the partial fraction expansion of the step response and the printed the results R, P, and K using the equation given to us in the lab handout. Figure 5 portrays the plot of the time-domain response from 0 ≤ t ≤ 4.5s using the cosine method. Figure 6 portrays the plot of Part 2 task 1 of this lab using sig.step.



\begin{figure}
\includegraphics[width=\linewidth]{Part1Task1Plot.png}
\caption{Hand Solved Step Response}
\end{figure}

\begin{figure}
\includegraphics[width=\linewidth]{Part1Task2Plot.png}
\caption{Step Response Using Sig.Step }
\end{figure}

\begin{figure}
\includegraphics[width=\linewidth]{Part1Task3Output.PNG}
\caption{Printed Partial Fraction Expansion}
\end{figure}

\begin{figure}
\includegraphics[width=\linewidth]{Part2Task1Output.PNG}
\caption{Printed Partial Fraction Expansion}
\end{figure}

\begin{figure}
\includegraphics[width=\linewidth]{Part2Task2Plot.png}
\caption{Time Domain Response}
\end{figure}

\begin{figure}
\includegraphics[width=\linewidth]{Part2Task3Plot.png}
\caption{H(s) Response Check}
\end{figure}



\newpage

\section{Error Analysis}

When working through this lab, I did experience one difficulty. At first, I struggled with printing my R and P values, but I was able to quickly fix my mistake and finish the lab. I also thought the prelab was good example of what we have been learning in class. 



\section{Questions}

\textbf{1. For a non-complex pole-residue term, you can still use the cosine method, explain why this works.}
{I believe this works because Cosine is not undefined at 0. Since the Sine method cannot be evaluated at zero, it makes sense that the Cosine method would be used when we don't have a complex pole residue. }

\textbf{2. Leave any feedback on the clarity of the expectations, instructions, and deliverables.}
{I think everything was clarified well in the Lab 6 document. I did not struggle with any of the expectations or instructions given in this lab. I thought the lab was easy to follow and complete. I also thought that the prelab was a useful exercise.  }



\section{Conclusion}

During this lab, I learned how to use the python function, sig.residue, to perform a partial fraction expansion. I felt that the lab was successful in its goal of furthering our understanding of partial fraction expansions. I don't have any recommendations to change this lab. I thought the lab provided all the information necessary for us to complete it. Here is the link to my github: \href{https://github.com/JMac1999}{Click Here}

\newpage


\begin{thebibliography}{111}

  \bibitem{ACMT}
A. Aldroubi, C. Cabrelli, U. Molter, and Sui Tang,
Dynamical sampling, 
{\it  Applied and Computational Harmonic Analysis}, doi:10.1016/j.acha.2015.08.014, 2016


\end{thebibliography}
\end{document}

