%%%%%%%%%%%%%%%%%%%%%%%%%%%%%%%%%%%%%%%%%%%%%%%%%%%%%%%%%%%%%%%%
%                                                              %
% Jared McDaniel                                               %
% ECE 351 Section 51                                           %
% Lab 11                                                       %
% 11/16/2021                                                   %
% This file contains the contents of the Lab 10 Report         %
%                                                              %
%%%%%%%%%%%%%%%%%%%%%%%%%%%%%%%%%%%%%%%%%%%%%%%%%%%%%%%%%%%%%%%%


\documentclass[12pt]{report}
\usepackage[english]{babel}
%\usepackage{natbib}
\usepackage{url}
\usepackage[utf8x]{inputenc}
\usepackage{amsmath}
\usepackage{graphicx}
\graphicspath{{images/}}
\usepackage{parskip}
\usepackage{fancyhdr}
\usepackage{vmargin}
\usepackage{listings}
\usepackage{hyperref}
\usepackage{xcolor}

\definecolor{codegreen}{rgb}{0,0.6,0}
\definecolor{codegray}{rgb}{0.5,0.5,0.5}
\definecolor{codeblue}{rgb}{0,0,0.95}
\definecolor{backcolour}{rgb}{0.95,0.95,0.92}

\lstdefinestyle{mystyle}{
    backgroundcolor=\color{backcolour},   
    commentstyle=\color{codegreen},
    keywordstyle=\color{codeblue},
    numberstyle=\tiny\color{codegray},
    stringstyle=\color{codegreen},
    basicstyle=\ttfamily\footnotesize,
    breakatwhitespace=false,         
    breaklines=true,                 
    captionpos=b,                    
    keepspaces=true,                 
    numbers=left,                    
    numbersep=5pt,                  
    showspaces=false,                
    showstringspaces=false,
    showtabs=false,                  
    tabsize=2
}
 
\lstset{style=mystyle}

\setmarginsrb{3 cm}{2.5 cm}{3 cm}{2.5 cm}{1 cm}{1.5 cm}{1 cm}{1.5 cm}

\title{Lab 11}								
% Title
\author{ Jared McDaniel}						
% Author
\date{11/16/2021}
% Date

\makeatletter
\let\thetitle\@title
\let\theauthor\@author
\let\thedate\@date
\makeatother

\pagestyle{fancy}
\fancyhf{}
\rhead{\theauthor}
\lhead{\thetitle}
\cfoot{\thepage}
%%%%%%%%%%%%%%%%%%%%%%%%%%%%%%%%%%%%%%%%%%%%
\begin{document}

%%%%%%%%%%%%%%%%%%%%%%%%%%%%%%%%%%%%%%%%%%%%%%%%%%%%%%%%%%%%%%%%%%%%%%%%%%%%%%%%%%%%%%%%%

\begin{titlepage}
	\centering
    \vspace*{0.5 cm}
   % \includegraphics[scale = 0.075]{bsulogo.png}\\[1.0 cm]	% University Logo
\begin{center}    \textsc{\Large   ECE 351 - Section 51 }\\[2.0 cm]	\end{center}% University Name
	\textsc{\Large Z-Transform Operations  }\\[0.5 cm]				% Course Code
	\rule{\linewidth}{0.2 mm} \\[0.4 cm]
	{ \huge \bfseries \thetitle}\\
	\rule{\linewidth}{0.2 mm} \\[1.5 cm]
	
	\begin{minipage}{0.4\textwidth}
		\begin{flushleft} \large
		%	\emph{Submitted To:}\\
		%	Name\\
          % Affiliation\\
           %contact info\\
			\end{flushleft}
			\end{minipage}~
			\begin{minipage}{0.4\textwidth}
            
			\begin{flushright} \large
			\emph{Submitted By :} \\
			Jared McDaniel  
		\end{flushright}
           
	\end{minipage}\\[2 cm]
	
%	\includegraphics[scale = 0.5]{PICMathLogo.png}
    
    
    
    
	
\end{titlepage}

%%%%%%%%%%%%%%%%%%%%%%%%%%%%%%%%%%%%%%%%%%%%%%%%%%%%%%%%%%%%%%%%%%%%%%%%%%%%%%%%%%%%%%%%%

\tableofcontents
\pagebreak

%%%%%%%%%%%%%%%%%%%%%%%%%%%%%%%%%%%%%%%%%%%%%%%%%%%%%%%%%%%%%%%%%%%%%%%%%%%%%%%%%%%%%%%%%
\renewcommand{\thesection}{\arabic{section}}
\section{Introduction}
 

The goal of this lab is to Analyze a discrete system using Python’s built-in functions and a function developed by Christopher Felton.
 
   

\section{Equations}
{The equations below portray the ones that were used in the lab. For this lab we were given one equation in the lab handout, and tasked to derive it to find the transfer function and the z-transform. Equation 1 depicts the equation given to us in the lab handout. Equation 2 depicts our transfer function. The residues and poles of equation 2 can be seen in Figure 1 under the Results section of this lab Report. The pole zero plot for this equation can be seen in Figure 2 under the Results section of this lab Report. The magnitude and phase response plot for this equation can be seen in Figure 3 under the results section of this lab report. Equation 3 depicts the z-transform of our transfer function. } 
 
\begin{equation}
    y[k] = 2x[k] − 40x[k−1] + 10y[k−1] − 16y[k−2]
\end{equation}
\begin{equation}
    H(z)  = \frac{2-40z^{-1}}{1 - 10z^{-1} - 16z^{-2}}
\end{equation}
\begin{equation}
    h[k] = -4(8^{k})u[k]+6(2^{k})u[k]
\end{equation}


\section{Methodology}

For Part 1 and Part 2 of this lab we were tasked with finding H(z) and h[k] by hand. The lab handout specifically asks us to find h[k] using partial fraction expansion. Below shows my steps to calculating both of these functions.  
\begin{equation}
    y[k] = 2x[k] − 40x[k−1] + 10y[k−1] − 16y[k−2]
\end{equation}
\begin{equation}
    Y(z)(1 - 10z^{-1} - 16z^{-2}) = X(z)(2-40z^{-1})
\end{equation}
\begin{equation}
    H(z)  = \frac{2-40z^{-1}}{1 - 10z^{-1} - 16z^{-2}}
\end{equation}
\begin{equation}
    \frac{H(z)}{z} = \frac{2(z-20)}{(z-8)*(z-2)}
\end{equation}
\begin{equation}
    \frac{H(z)}{z} = \frac{A}{(z-8)}+\frac{B}{z-2}
\end{equation}
\begin{equation}
    A = -4
\end{equation}
\begin{equation}
    B = 6
\end{equation}
\begin{equation}
    H(z)  = \frac{-4z}{z-8} + \frac{6z}{z-2}
\end{equation}
\begin{equation}
    h[k] = -4(8^{k})u[k]+6(2^{k})u[k]
\end{equation}

{This is my code for Part 1 Task 3, and 4 of Lab 11. For the Part 1 Task 3 code, we were tasked with using the sig.residue() function to confirm our partial fraction expansion that we did in Part 2 of this lab. To do this I defined the numerator and the denominator of my transfer function. Then I put those in the sig.residue() function and printed the residue, pole, and K values. For the Part 1 Task 4 code, we were tasked with using the zplane() function to obtain the pole-zero plot of the transfer function. To do this, I inputted my numerator and denominator from Part 3 of this lab into the zplane() function. The residues and poles from the part 3 code can be found in Figure 1 under the Results section of this lab report. The pole-zero plot can be found in Figure 2 under the Results section of this lab report.}
\begin{lstlisting}[language=Python]
#Part 3

num = [2, -40]
den = [1, -10, 16]

r, p, k = sig.residuez(num, den)

print('Residues = ', r, 'Poles = ', p, 'K = ', k)

#Part 4

z, p, k = zplane(num, den) #obtains pole-zero plot for H(z)
\end{lstlisting}

{This is my code for Part 1 Task 5 of Lab 11. For this section of code we were tasked with using the sig.freqz() function to plot the magnitude and phase response of our transfer function. To do this, I inputted my numerator, denominator, worN=512, and whole = false into the sig.freqz() function. Then I defined my magnitude and phase response of H(z). Then I used the plotting code that we have been using all semester to create the plots. For the magnitude plot I made the units decibels, and for the phase plot I used radians as my unit. These two plots can be found in Figure 3 under the Results section of this lab report.  }
\begin{lstlisting}[language=Python]
#Part 5

w, h = sig.freqz(num, den, worN=512, whole = True)
hmag = np.abs(h)
hangle = np.angle(h)

plt.figure(figsize=(10,7))
plt.subplot(2,1,1)
plt.plot(w, 20*np.log10(hmag))
plt.grid()
plt.title("Part 5 Plot")
plt.ylabel("Frequency Response")


plt.subplot(2,1,2)
plt.plot(w/np.pi, 180/np.pi*hangle)
plt.grid()
plt.xlabel("Frequency")
plt.ylabel("Frequency Response") 
\end{lstlisting}


\newpage
\section{Results}

The results for this lab are shown below as plots and printed values. Figure 1 shows the residue and pole values of our transfer function. Figure 2 shows the pole-zero plot for our transfer function. Figure 3 portrays two plots. The top plot is the magnitude response of our transfer function. The bottom plot is the phase response of our transfer function. 


\begin{figure}
\includegraphics[width=\linewidth]{Poles&Residues.PNG}
\caption{Residues and Poles of H(z)}
\end{figure}



\begin{figure}
\includegraphics[width=\linewidth]{Plot 1.png}
\caption{Pole-Zero Plot of H(z)}
\end{figure}



\begin{figure}
\includegraphics[width=\linewidth]{Plot 2.png}
\caption{Magnitude and Phase Response of H(z)}
\end{figure}



\newpage

\section{Error Analysis}

When working through this lab, I did not experience any difficulties. I thought the lab provided clear explanations that helped me complete it without any issues. I also thought the code given in the lab handout was useful. Without that, I'm not sure how I would've created the second Figure in this lab report. 



\section{Questions}

\textbf{1. Looking at the plot generated in Task 4, is H(z) stable? Explain why or why not.}
{From the plot generated in Task 4, I think H(z) is unstable. According to our class text, all roots (poles) from our characteristic equation must be inside the unit circle to be considered stable. This is not the case, so H(z) is unstable.   }


\textbf{4. Leave any feedback on the clarity of the expectations, instructions, and deliverables.}
{I think everything was clarified well in the Lab 11 document. I thought the code given in the lab handout was useful in creating my first plot. I also thought the lab was easy to follow and complete.}



\section{Conclusion}

During this lab, I became more familiar with analyzing a discrete system using Python's built-in functions. I felt that the lab was successful in its goal of furthering our understanding of z-transforms. I don't have any recommendations to change this lab. I thought the lab provided all the necessary information for us to complete it. Here is the link to my github: \href{https://github.com/JMac1999}{Click Here}




\begin{thebibliography}{111}

  \bibitem{ACMT}
A. Aldroubi, C. Cabrelli, U. Molter, and Sui Tang,
Dynamical sampling, 
{\it  Applied and Computational Harmonic Analysis}, doi:10.1016/j.acha.2015.08.014, 2016


\end{thebibliography}
\end{document}

