%%%%%%%%%%%%%%%%%%%%%%%%%%%%%%%%%%%%%%%%%%%%%%%%%%%%%%%%%%%%%%%%
%                                                              %
% Jared McDaniel                                               %
% ECE 351 Section 51                                           %
% Lab 4                                                        %
% 9/28/2021                                                    %
% This file contains the contents of the Lab 4 Report          %
%                                                              %
%%%%%%%%%%%%%%%%%%%%%%%%%%%%%%%%%%%%%%%%%%%%%%%%%%%%%%%%%%%%%%%%


\documentclass[12pt]{report}
\usepackage[english]{babel}
%\usepackage{natbib}
\usepackage{url}
\usepackage[utf8x]{inputenc}
\usepackage{amsmath}
\usepackage{graphicx}
\graphicspath{{images/}}
\usepackage{parskip}
\usepackage{fancyhdr}
\usepackage{vmargin}
\usepackage{listings}
\usepackage{hyperref}
\usepackage{xcolor}

\definecolor{codegreen}{rgb}{0,0.6,0}
\definecolor{codegray}{rgb}{0.5,0.5,0.5}
\definecolor{codeblue}{rgb}{0,0,0.95}
\definecolor{backcolour}{rgb}{0.95,0.95,0.92}

\lstdefinestyle{mystyle}{
    backgroundcolor=\color{backcolour},   
    commentstyle=\color{codegreen},
    keywordstyle=\color{codeblue},
    numberstyle=\tiny\color{codegray},
    stringstyle=\color{codegreen},
    basicstyle=\ttfamily\footnotesize,
    breakatwhitespace=false,         
    breaklines=true,                 
    captionpos=b,                    
    keepspaces=true,                 
    numbers=left,                    
    numbersep=5pt,                  
    showspaces=false,                
    showstringspaces=false,
    showtabs=false,                  
    tabsize=2
}
 
\lstset{style=mystyle}

\setmarginsrb{3 cm}{2.5 cm}{3 cm}{2.5 cm}{1 cm}{1.5 cm}{1 cm}{1.5 cm}

\title{Lab 4}								
% Title
\author{ Jared McDaniel}						
% Author
\date{9/28/2021}
% Date

\makeatletter
\let\thetitle\@title
\let\theauthor\@author
\let\thedate\@date
\makeatother

\pagestyle{fancy}
\fancyhf{}
\rhead{\theauthor}
\lhead{\thetitle}
\cfoot{\thepage}
%%%%%%%%%%%%%%%%%%%%%%%%%%%%%%%%%%%%%%%%%%%%
\begin{document}

%%%%%%%%%%%%%%%%%%%%%%%%%%%%%%%%%%%%%%%%%%%%%%%%%%%%%%%%%%%%%%%%%%%%%%%%%%%%%%%%%%%%%%%%%

\begin{titlepage}
	\centering
    \vspace*{0.5 cm}
   % \includegraphics[scale = 0.075]{bsulogo.png}\\[1.0 cm]	% University Logo
\begin{center}    \textsc{\Large   ECE 351 - Section 51 }\\[2.0 cm]	\end{center}% University Name
	\textsc{\Large System Step Response Using Convolution  }\\[0.5 cm]				% Course Code
	\rule{\linewidth}{0.2 mm} \\[0.4 cm]
	{ \huge \bfseries \thetitle}\\
	\rule{\linewidth}{0.2 mm} \\[1.5 cm]
	
	\begin{minipage}{0.4\textwidth}
		\begin{flushleft} \large
		%	\emph{Submitted To:}\\
		%	Name\\
          % Affiliation\\
           %contact info\\
			\end{flushleft}
			\end{minipage}~
			\begin{minipage}{0.4\textwidth}
            
			\begin{flushright} \large
			\emph{Submitted By :} \\
			Jared McDaniel  
		\end{flushright}
           
	\end{minipage}\\[2 cm]
	
%	\includegraphics[scale = 0.5]{PICMathLogo.png}
    
    
    
    
	
\end{titlepage}

%%%%%%%%%%%%%%%%%%%%%%%%%%%%%%%%%%%%%%%%%%%%%%%%%%%%%%%%%%%%%%%%%%%%%%%%%%%%%%%%%%%%%%%%%

\tableofcontents
\pagebreak

%%%%%%%%%%%%%%%%%%%%%%%%%%%%%%%%%%%%%%%%%%%%%%%%%%%%%%%%%%%%%%%%%%%%%%%%%%%%%%%%%%%%%%%%%
\renewcommand{\thesection}{\arabic{section}}
\section{Introduction}
 

The goal of this lab is to become more familiar with  using convolution to compute a system’s step response.

\section{Equations}

The equations below portray the ones that were used during this lab. In Part 1 of this lab, the h(t) functions were plotted as a single figure. In Part 2 of this lab, the h(t) functions were convolved. We convolved functions h1(t) with u(t), h2(t) with u(t), and h3(t) with u(t). Also in Part 2 of this lab, we were tasked with integrating and plotting h1(t), h2(t), and h3(t). The y(t) equations are the integrated forms of the h(t) equations.

\begin{equation}
    h_{1}(t) = e^{-2t}[u(t) - u(t-3)]
\end{equation}
\begin{equation}
    h_{2}(t) = u(t−2) − u(t−6)
\end{equation}
\begin{equation}
    h_{3}(t) = cos(w_{0}t)u(t)
\end{equation}
\begin{equation}
    y_{1}(t) = \frac{1}{2}(1-e^{-2t})u(t) - \frac{1}{2}(1-e^{-2(t-3)})u(t-3)
\end{equation}
\begin{equation}
    y_{2}(t) = (t-2)u(t-2) - (t-6)u(t-6)
\end{equation}
\begin{equation}
    y_{3}(t) = \frac{1}{w_{0}}sin(w_{0}t)u(t)
\end{equation}


\section{Methodology}

This is my code for Part 1 of Lab 4. I started this section of code by copying my defined step function from Lab 3. Then I created the three user-defined functions given to us in the Lab 4 hangout. Then I plotted each function using subplots. The plots for the three user-defined functions can be viewed in Figure 1 under the Results section of this lab report.  
\begin{lstlisting}[language=Python]
#%% Part 1 Code

import numpy as np
import matplotlib.pyplot as plt
import scipy.signal as sig

plt.rcParams.update({ 'font.size' :14}) #font size on plots

steps = 1e-2 #step size
t = np.arange(-10, 10 + steps, steps) 
#plot starts at zero and finishes at 10 on the x-axis

def u(t): #step function
    y=np.zeros(t.shape) #y(t) is initialized as an array of zeros
    
    for i in range(len(t)):
        if(t[i]<0):
            y[i]=0 #step function is 0 when t < 0
        else:
            y[i]=1 #step function is 1 when t >= 0
    return y #return outputs stored in the array

f = 0.25
w = 2*np.pi*f

def h1(t): return np.exp(-2*t)*(u(t)-u(t-3))
def h2(t): return u(t-2)-u(t-6)
def h3(t): return np.cos(w*t)*u(t)

plt.figure(figsize = (10,7))
#f1(t) plot
plt.subplot(3, 1, 1) #subplot code taken from lab 1 handout. Subplot 1
plt.plot(t, h1(t))
plt.grid()
plt.ylabel('h1(t)')
plt.title('Part 1 Functions')

#f2(t) plot
plt.subplot(3, 1, 2) #subplot code taken from lab 1 handout. Subplot 2
plt.plot(t, h2(t))
plt.grid()
plt.ylabel('h2(t)')

#f3(t) plot
plt.subplot(3, 1, 3) #subplot code taken from lab 1 handout. Subplot 3
plt.plot(t, h3(t))
plt.grid()
plt.ylabel('h3(t)')
plt.xlabel('t') # xlabel only needs to go on the bottom plot
plt.show()
\end{lstlisting}

{This is my code for Part 2 Task 1 of Lab 4. For this section of the lab, we were tasked with plotting a step response using the convolution function I created in Lab 3. To do this, I first copied the convolution code from Lab 3. Then I set my h(t) functions equal to three different variables. Then I plotted each function using subplots. The plots for these convolutions can be viewed in Figure 2 under the Results section of this lab report. }
\begin{lstlisting}[language=Python]
#%% Part 2 Task 1 Code

def c(f1, f2): #defining convolution in python for f1 and f2.
    n1 = len(f1) #returns the number of elements in an array. (horizontal)
                 #returns an integer representing the items in f1 passed as
                 #an argument.
                 #length of array
    n2 = len(f2) #returns the number of elements in an array. (horizontal)
                 #returns an integer representing the items in f1 passed as
                 #an argument.
                 #length of array
    l1 = np.append(f1 , np.zeros((1, n2-1)))
    # np.append adds values to the end of a numpy array.
    # values are appended to a copy of this f1 array.
    # np.zeros defines an array of zeros. In this case, 1 is the shape of the
    # array and n2-1 is the desired data type of the array
    #causes f1 to be the same as f2
    l2 = np.append(f2 , np.zeros((1, n1-1)))
    # np.append adds values to the end of a numpy array.
    # values are appended to a copy of this f2 array.
    # np.zeros defines an array of zeros. In this case, 1 is the shape of the
    # array and n2-1 is the desired data type of the array
    #causes f2 to be the same as f1
    result = np.zeros(l1.shape) #creates an array of zeros with the l1 shape
    
    for i in range(n2+n1-2): #range creates a range of numbers that is nice
                             #for for loops
                             #adds n2 and n1((n2-1) + (n1-1))
                             #i goes through both
        result[i] = 0 #initializes all of them to be zero
        for j in range(n1): #when j is in the range of n1, if i-j+1 is > 0
                            #result[i] will ne added to the right side of the 
                            #equation.
            if (i - j + 1 > 0): 
            #if length of both functions is greater than the first function
                try:
                    result[i] += l1[j]*l2[i-j+1] 
                    #adds duration of each function
                except:
                    print(i,j)
    return result*steps #returns result value

N = len(t)  #returns the number of elements in an array. (horizontal)
            #returns an integer representing the items in t passed as an
            #argument.
t1 = np.arange(2*t[0], 2*t[N-1]+steps, steps)
#create a numpy array that is a range. My interval starts at
# 0 and goes to 2*t[N-1]. The steps are values at 1e-2.    
f1=h1(t) #setting f1 = to my f1(t) function
f2=h2(t) #setting f2 = to my f2(t) function
f3=h3(t) #setting f3 = to my f3(t) function

plt.figure(figsize = (10,7))
#f1(t) plot
plt.subplot(3, 1, 1) #subplot code taken from lab 1 handout. Subplot 1
plt.plot(t1, c(f1,u(t)))
plt.ylabel('h1(t) & u(t)')
plt.title('Part 2 Task 1 Convolutions')
plt.xlim(-10,20)
plt.grid()

#f2(t) plot
plt.subplot(3, 1, 2) #subplot code taken from lab 1 handout. Subplot 2
plt.plot(t1, c(f2,u(t)))
plt.xlim(-10,10)
plt.grid()
plt.ylabel('h2(t) & u(t)')

#f3(t) plot
plt.subplot(3, 1, 3) #subplot code taken from lab 1 handout. Subplot 3
plt.plot(t1, c(f3,u(t)))
plt.xlim(-10,10)
plt.grid()
plt.ylabel('h1(t) & u(t)')
plt.xlabel('t') # xlabel only needs to go on the bottom plot
plt.show()
\end{lstlisting}



{This is my code for Part 2 Task 2 of Lab 4. In this part of the lab we were tasked with integrating the h(t) functions given to us in the lab handout. Once that was completed, I inserted each of those functions into my code. Then I plotted each function using subplots. The plots can be viewed in Figure 3 under the Results section of this lab report. }
\begin{lstlisting}[language=Python]
#%% Part 2 Task 2 Code

y1 = ((1/2)*(1-np.exp(-2*t))*u(t))-((1/2)*(1-np.exp(-2*(t-3)))*u(t-3))
y2 = (t-2)*u(t-2)-(t-6)*u(t-6)
y3 = (1/w)*np.sin(w*t)*u(t)

plt.figure(figsize = (10,7))
#f1(t) plot
plt.subplot(3, 1, 1) #subplot code taken from lab 1 handout. Subplot 1
plt.plot(t, y1)
plt.grid()
plt.xlim(-10,20)
plt.ylabel('y1(t)')
plt.title('Part 2 Hand Calculated Functions')

#f2(t) plot
plt.subplot(3, 1, 2) #subplot code taken from lab 1 handout. Subplot 2
plt.plot(t, y2)
plt.grid()
plt.ylabel('y2(t)')

#f3(t) plot
plt.subplot(3, 1, 3) #subplot code taken from lab 1 handout. Subplot 3
plt.plot(t, y3)
plt.grid()
plt.ylabel('y3(t)')
plt.xlabel('t') # xlabel only needs to go on the bottom plot
plt.show()
\end{lstlisting}



\section{Results}

The results for this lab are shown below as plots. Figure 1 portrays the plots of the three h(t) functions given to us in the lab handout. Figure 2 portrays the step response plots of the three h(t) functions. Figure 3 portrays the plots of my hand calculated integrals of the three h(t) functions.    



\begin{figure}
\includegraphics[width=\linewidth]{Part1Functions.png}
\caption{Part 1 User-Defined Functions}
\end{figure}

\begin{figure}
\includegraphics[width=\linewidth]{Part2Convolutions.png}
\caption{Step Response}
\end{figure}

\begin{figure}
\includegraphics[width=\linewidth]{Part2HandCalcs.png}
\caption{Hand Calculated Integrals}
\end{figure}



\newpage

\section{Error Analysis}

When working through this lab, I did not experience any difficulties. I did initially compute my first hand calculation incorrectly, but I was able to find my error and fix my mistake. 



\section{Questions}

\textbf{1. Leave any feedback on the clarity of the expectations, instructions, and deliverables.}
{I think everything was clarified well in the lab 4 document. I did not struggle with any of the expectations or instructions given in this lab. I thought the lab was easy to follow and complete. }



\section{Conclusion}

During this lab, I learned how to use convolutions to compute the step response of the system. I felt that the lab was successful in its goal of furthering our understanding of convolutions. I don't have any recommendations to change this lab. I thought it was straightforward, and I thought it provided a deeper insight of using convolutions to compute the step response. Here is the link to my github: \href{https://github.com/JMac1999}{Click Here}

\newpage


\begin{thebibliography}{111}

  \bibitem{ACMT}
A. Aldroubi, C. Cabrelli, U. Molter, and Sui Tang,
Dynamical sampling, 
{\it  Applied and Computational Harmonic Analysis}, doi:10.1016/j.acha.2015.08.014, 2016


\end{thebibliography}
\end{document}

