%%%%%%%%%%%%%%%%%%%%%%%%%%%%%%%%%%%%%%%%%%%%%%%%%%%%%%%%%%%%%%%%
%                                                              %
% Jared McDaniel                                               %
% ECE 351 Section 51                                           %
% Lab 12                                                       %
% 12/7/2021                                                    %
% This file contains the contents of the Lab 12 Report         %
%                                                              %
%%%%%%%%%%%%%%%%%%%%%%%%%%%%%%%%%%%%%%%%%%%%%%%%%%%%%%%%%%%%%%%%


\documentclass[12pt]{report}
\usepackage[english]{babel}
%\usepackage{natbib}
\usepackage{url}
\usepackage[utf8x]{inputenc}
\usepackage{amsmath}
\usepackage{graphicx}
\graphicspath{{images/}}
\usepackage{parskip}
\usepackage{fancyhdr}
\usepackage{vmargin}
\usepackage{listings}
\usepackage{hyperref}
\usepackage{xcolor}

\definecolor{codegreen}{rgb}{0,0.6,0}
\definecolor{codegray}{rgb}{0.5,0.5,0.5}
\definecolor{codeblue}{rgb}{0,0,0.95}
\definecolor{backcolour}{rgb}{0.95,0.95,0.92}

\lstdefinestyle{mystyle}{
    backgroundcolor=\color{backcolour},   
    commentstyle=\color{codegreen},
    keywordstyle=\color{codeblue},
    numberstyle=\tiny\color{codegray},
    stringstyle=\color{codegreen},
    basicstyle=\ttfamily\footnotesize,
    breakatwhitespace=false,         
    breaklines=true,                 
    captionpos=b,                    
    keepspaces=true,                 
    numbers=left,                    
    numbersep=5pt,                  
    showspaces=false,                
    showstringspaces=false,
    showtabs=false,                  
    tabsize=2
}
 
\lstset{style=mystyle}

\setmarginsrb{3 cm}{2.5 cm}{3 cm}{2.5 cm}{1 cm}{1.5 cm}{1 cm}{1.5 cm}

\title{Lab 12}								
% Title
\author{ Jared McDaniel}						
% Author
\date{12/7/2021}
% Date

\makeatletter
\let\thetitle\@title
\let\theauthor\@author
\let\thedate\@date
\makeatother

\pagestyle{fancy}
\fancyhf{}
\rhead{\theauthor}
\lhead{\thetitle}
\cfoot{\thepage}
%%%%%%%%%%%%%%%%%%%%%%%%%%%%%%%%%%%%%%%%%%%%
\begin{document}

%%%%%%%%%%%%%%%%%%%%%%%%%%%%%%%%%%%%%%%%%%%%%%%%%%%%%%%%%%%%%%%%%%%%%%%%%%%%%%%%%%%%%%%%%

\begin{titlepage}
	\centering
    \vspace*{0.5 cm}
   % \includegraphics[scale = 0.075]{bsulogo.png}\\[1.0 cm]	% University Logo
\begin{center}    \textsc{\Large   ECE 351 - Section 51 }\\[2.0 cm]	\end{center}% University Name
	\textsc{\Large Filter Design  }\\[0.5 cm]				% Course Code
	\rule{\linewidth}{0.2 mm} \\[0.4 cm]
	{ \huge \bfseries \thetitle}\\
	\rule{\linewidth}{0.2 mm} \\[1.5 cm]
	
	\begin{minipage}{0.4\textwidth}
		\begin{flushleft} \large
		%	\emph{Submitted To:}\\
		%	Name\\
          % Affiliation\\
           %contact info\\
			\end{flushleft}
			\end{minipage}~
			\begin{minipage}{0.4\textwidth}
            
			\begin{flushright} \large
			\emph{Submitted By :} \\
			Jared McDaniel  
		\end{flushright}
           
	\end{minipage}\\[2 cm]
	
%	\includegraphics[scale = 0.5]{PICMathLogo.png}
    
    
    
    
	
\end{titlepage}

%%%%%%%%%%%%%%%%%%%%%%%%%%%%%%%%%%%%%%%%%%%%%%%%%%%%%%%%%%%%%%%%%%%%%%%%%%%%%%%%%%%%%%%%%

\tableofcontents
\pagebreak

%%%%%%%%%%%%%%%%%%%%%%%%%%%%%%%%%%%%%%%%%%%%%%%%%%%%%%%%%%%%%%%%%%%%%%%%%%%%%%%%%%%%%%%%%
\renewcommand{\thesection}{\arabic{section}}
\section{Introduction}
 

The goal of this lab is to apply the skills and concepts from the course into a practical application and present the information in a professional report. 
 
   

\section{Equations}
{The equation below portrays the one that was used in the lab. For this lab the only equation we used was the transfer function of an analog filter circuit. The filter itself was a bandpass filter. From this transfer function, we created four bode plots. Figure 6 is the entire bode plot of the analog filter circuit. Figure 7 shows the position measurement information attenuated by less than -0.3dB. Figure 8 shows the low frequency vibration noise attenuated by at least -30dB. Figure 9 shows the switching amplifier noise attenuated by at least -21dB. Figures 6, 7, 8, and 9 can be seen under the Results section of this lab report. } 
 
\begin{equation}
    H(s) = \frac{\frac{1}{RC}s}{s^{2}+\frac{1}{RC}s+\frac{1}{LC}}
\end{equation}


\section{Methodology}

For Part 1 of this lab we were tasked with identifying the noise magnitudes and corresponding frequencies due to the low frequency vibration and switching amplifier. We were also tasked with finding the magnitudes and corresponding frequencies of the position measurement information. To do this, I started by copying the starter code given to us in the appendix of the lab handout. Then I defined my sampling frequency. After my sampling frequency was defined I inputted that frequency and the sensor signal through my defined clean fast Fourier function. I obtained this function during lab 9 for this class. Finally I generated five plots using the plotting code from previous labs, and the defined stem() function given to us in the lab handout. Figure 1 shows the noisy input signal that we were given in the lab handout. Figure 2 shows the noisy input signal after it was ran through the fast Fourier transform. Figure 3 portrays the low frequency vibration noise after it was ran through the fast Fourier transform. The frequency value is about 60Hz with a magnitude of about 0.8. Figure 4 portrays the position measurement signal after it was ran through the fast Fourier transform. The frequencies are between 1800 and 2000 Hz, with varying magnitudes. Figure 5 portrays the high frequency noise after it was ran through the fast Fourier transform. The frequencies are right around 50,000 Hz with magnitudes ranging from 0.5 to 0.8. These first five figures can be seen under the Results section of this lab report. 
\begin{lstlisting}[language=Python]
#Noisy Signal Starter Code from the lab handout

#load input signal
df = pd.read_csv('NoisySignal.csv')

t = df['0'].values
sensor_sig = df['1'].values

plt.figure(figsize = (10,7))
plt.plot(t, sensor_sig)
plt.grid()
plt.title('Noisy Input Signal')
plt.xlabel('Time [s]')
plt.ylabel('Amplitude [V]')
plt.show()


fs = 1/(t[1]-t[0])
X_mag, X_phi, freq = CleanFastFour(sensor_sig, fs) #code taken from lab 9


#to use a function as a single plot
fig, ax = plt.subplots(figsize =(10, 7)) #from 6.1.2 in the lab handout
make_stem(ax, freq, X_mag) #form 6.1.2 in the lab handout
plt.title("CFFT for the Noisy Input Signal")
plt.xlabel('Frequency (Hz)')
plt.ylabel('Magnitude (dB)')
plt.show()

#to use a function as a single plot
fig, ax = plt.subplots(figsize =(10, 7)) #from 6.1.2 in the lab handout
make_stem(ax, freq, X_mag) #form 6.1.2 in the lab handout
plt.xlim(0,400) 
plt.title("CFFT for the Low Frequency Vibration Noise")
plt.xlabel('Frequency (Hz)')
plt.ylabel('Magnitude (dB)')
plt.show()

#to use a function as a single plot
fig, ax = plt.subplots(figsize =(10, 7)) #from 6.1.2 in the lab handout
make_stem(ax, freq, X_mag) #form 6.1.2 in the lab handout
plt.xlim(17.5e2,20.5e2) #these change as my R, L, & C values change
plt.title("CFFT for the Position Measurement Signal")
plt.xlabel('Frequency (Hz)')
plt.ylabel('Magnitude (dB)')
plt.show()

#to use a function as a single plot
fig, ax = plt.subplots(figsize =(10, 7)) #from 6.1.2 in the lab handout
make_stem(ax, freq, X_mag) #form 6.1.2 in the lab handout
plt.xlim(4e4,6e4) #these change as my R, L, & C values change
plt.title("CFFT for the High Frequency Noise")
plt.xlabel('Frequency (Hz)')
plt.ylabel('Magnitude (dB)')
plt.show()

\end{lstlisting}

{This is my code for Task 2 of Lab 12. For this section of the lab we were tasked with designing an analog filter circuit to remove the noise and only pass the position measurement. The analog filter circuit is a bandpass filter similar to the one I created in lab 10. I started by defining my step and omega values. Then I defined the transfer function using the same code from lab 10. Then I generated my bode plot using the con.TrasnferFunction() and the con.bode() functions from lab 10. Finally, I generated 3 other plots that portrayed position measurement information, low-frequency vibration noise, and switching amplifier noise. I did this by using the plotting code and the con.bode() function. Figure 6 portrays my bode plot for this lab. Figure 7 portrays the position measurement plot between 1.8 and 2 kHz attenuated by less than -0.3dB. Figure 8 portrays the low-frequency vibration noise attenuated by at least -30dB. Figure 9 portrays the switching amplifier noise attenuated by at least -21dB. We also had to create a professional circuit schematic portraying our R, L, and C values of our filter design. I started by choosing a typical capacitor value of 5.1nF, and then I kept changing the values of my resistor and inductor until I obtained proper plots for Figures 6, 7, 8, and 9. My resistor value is 25k ohms and my inductor value is 1.27 Henries. Figure 10 portrays my schematic with the R, L, and C values I used. Figures 6, 7, 8, 9, and 10 can be seen under the Results section of this lab report.}
\begin{lstlisting}[language=Python]
#Bandpass Filter Code

steps = 10 #small step size
w = np.arange(10, 1e6 + steps, steps) 
#omega starts at 10 rad/s and finishes at 10^6 rad/s

#I need to change these values
R = 25000
L = 127e-2
C = 5.1e-9 #typical capacitor value I found online
#R, L, C values of our RLC circuit

#RLC transfer function from lab 10
num = np.array ([1/(R*C), 0]) #numerator of the transfer function
den = np.array ([1, 1/(R*C), 1/(L*C)]) #denominator of the transfer function

plt.figure(figsize=(10,7))
#lab 10 frequency response example code
sys = con.TransferFunction(num,den)
_ = con.bode(sys, w, dB = True, Hz = True, deg = True, plot = True)
#use _= to supress the output

plt.figure(figsize=(10,7))
_ = con.bode(sys, w, dB = True, Hz = True, deg = True, plot = True)
#plots won't show up unless I have con.bode()
plt.xlim(17.5e2,20.5e2) #these change as my R, L, & C values change
#plt.ylim(0, -0.3)
plt.title("Position Measurement Information")


plt.figure(figsize=(10,7))
_ = con.bode(sys, w, dB = True, Hz = True, deg = True, plot = True)
#plots won't show up unless I have con.bode()
plt.xlim(4e4,6e4) #these change as my R, L, & C values change
plt.title("Low-Frequency Vibration Noise")

plt.figure(figsize=(10,7))
_ = con.bode(sys, w, dB = True, Hz = True, deg = True, plot = True)
#plots won't show up unless I have con.bode()
plt.xlim(300,400) #these change as my R, L, & C values change
plt.title("Switching Amplifier Noise")
\end{lstlisting}

{This is my code for Task 3 of Lab 12. For this section of code we were tasked with using Python to filter the noisy sensor signal using the filter I designed. Then we had to demonstrate that the output signal (which is now filtered) has been correctly attenuated at the appropriate frequencies. I started this section of code by using the sig.bilinear() and sig.lfilter() functions from lab. These two functions pass the noisy signal through the circuit first, and through the filter after. Then I used the plotting code from previous labs to generate my filtered noisy signal. Once the signal was filtered, I ran it through my clean fast Fourier function again to obtain the filtered versions of the first five figures. To do this I used the same plotting code from part 1 of this lab. Figure 11 portrays my filtered input signal. Figure 12 portrays the filtered signal ran through the clean fast Fourier transform. Figure 13 portrays the filtered low frequency vibration noise ran through the clean fast Fourier transform. Figure 14 portrays the filtered position measurement signal ran through the clean fast Fourier transform. Figure 15 shows the filtered high frequency noise ran through the clean fast Fourier transform. Figures 11, 12, 13, 14, and 15 can be seen under the Results section of this lab report. }
\begin{lstlisting}[language=Python]
#Filtered Noisy Signal Code
#sig.bilinear and sig.lfilter was used in lab 10. 
#I copied it over into this lab

#convert the transfer function to the z-domain
#need to do this to pass the noisy signal through the RLC circuit
zeq, peq = sig.bilinear(num, den, fs=fs)  
#sig.lfilter is used to pass the noisy signal through the filter
y = sig.lfilter(zeq, peq, sensor_sig)

plt.figure(figsize=(10,7))
plt.plot(t, y)
plt.grid()
plt.title("Filtered Noisy Input Signal")
plt.xlabel("Time [s]")
plt.ylabel("Amplitude [V]") 



X_mag, X_phi, freq = CleanFastFour(y, fs) #code taken from lab 9

#to use a function as a single plot
fig, ax = plt.subplots(figsize =(10, 7)) #from 6.1.2 in the lab handout
make_stem(ax, freq, X_mag) #form 6.1.2 in the lab handout
plt.title("CFFT for the Filtered Noisy Input Signal")
plt.xlabel('Frequency (Hz)')
plt.ylabel('Magnitude (dB)')
plt.show()

#to use a function as a single plot
fig, ax = plt.subplots(figsize =(10, 7)) #from 6.1.2 in the lab handout
make_stem(ax, freq, X_mag) #form 6.1.2 in the lab handout
plt.xlim(0,400)
plt.title("CFFT for the Filtered Low Frequency Vibration Noise")
plt.xlabel('Frequency (Hz)')
plt.ylabel('Magnitude (dB)')
plt.show()

#to use a function as a single plot
fig, ax = plt.subplots(figsize =(10, 7)) #from 6.1.2 in the lab handout
make_stem(ax, freq, X_mag) #form 6.1.2 in the lab handout
plt.xlim(17.5e2,20.5e2) #these change as my R, L, & C values change
plt.title("CFFT for the Filtered Position Measurement Signal")
plt.xlabel('Frequency (Hz)')
plt.ylabel('Magnitude (dB)')
plt.show()

#to use a function as a single plot
fig, ax = plt.subplots(figsize =(10, 7)) #from 6.1.2 in the lab handout
make_stem(ax, freq, X_mag) #form 6.1.2 in the lab handout
plt.xlim(4e4,6e4) #these change as my R, L, & C values change
plt.title("CFFT for the Filtered High Frequency Noise")
plt.xlabel('Frequency (Hz)')
plt.ylabel('Magnitude (dB)')
plt.show()  
\end{lstlisting}
\newpage

\section{Results}

The results for this lab are shown below as plots. Figure 1 shows the noisy signal given to us in the lab handout. Figure 2 shows the noisy input signal after it was ran through the clean fast Fourier function. Figure 3 portrays the low frequency vibration noise after it was ran through the clean fast Fourier function. Figure 4 portrays the position measurement signal after it was ran through the clean fast Fourier function. Figure 5 shows the high frequency noise after it was ran through the clean fast Fourier function. Figures 6, 7, 8, and 9 each have two plots. Figure 6 shows the generated bode plot for this lab. Figure 7 portrays the position measurement plot within the desired frequency range and attenuated at the desired decibels. Figure 8 shows the low-frequency vibration noise attenuated at the desired decibels. Figure 9 portrays the switching amplifier noise attenuated at the desired decibels. Figure 10 portrays my circuit schematic with the desired R, L and C values I used for this lab. Figures 11, 12, 13, 14, 15 are the filtered version of Figures 1, 2, 3, 4, and 5. Figure 11 portrays the filtered input signal. Figure 12 portrays the filtered signal after it was ran through the clean fast Fourier function. Figure 13 portrays the filtered low frequency vibration noise after it was ran through the clean fast Fourier function. Figure 14 portrays the filtered position measurement signal after it was ran through the clean fast Fourier function. Figure 15 shows the filtered high frequency noise after it was ran through the clean fast Fourier function. Specific data points for each all of the plots are detailed in the Methodology section of this lab report. 



\begin{figure}
\includegraphics[width=\linewidth]{NoisyInputSignal.png}
\caption{Noisy Input Signal}
\end{figure}


\begin{figure}
\includegraphics[width=\linewidth]{CFFTNoisySignal.png}
\caption{CFFT Noisy Input Signal}
\end{figure}


\begin{figure}
\includegraphics[width=\linewidth]{CFFTLowFreqVibrationNoise.png}
\caption{CFFT Low Frequency Vibration Noise}
\end{figure}

\begin{figure}
\includegraphics[width=\linewidth]{CFFTPositionMeasurement.png}
\caption{CFFT Position Measurement Signal}
\end{figure}

\begin{figure}
\includegraphics[width=\linewidth]{CFFTHighFreqNoise.png}
\caption{CFFT High Frequency Noise}
\end{figure}

\begin{figure}
\includegraphics[width=\linewidth]{BodePlot.png}
\caption{Generated Bode Plot for the Designed Circuit}
\end{figure}

\begin{figure}
\includegraphics[width=\linewidth]{PositionMeasurementBodePlot.png}
\caption{Position Measurement Bode Plot}
\end{figure}

\begin{figure}
\includegraphics[width=\linewidth]{LowFreqVibrationNoiseBodePlot.png}
\caption{Low Frequency Vibration Noise Bode Plot}
\end{figure}

\begin{figure}
\includegraphics[width=\linewidth]{SwitchingAmplifierNoiseBodePlot.png}
\caption{Switching Amplifier Noise Bode Plot}
\end{figure}

\begin{figure}
\includegraphics[width=\linewidth]{CircuitSchematic.PNG}
\caption{Analog Filter Design}
\end{figure}

\begin{figure}
\includegraphics[width=\linewidth]{FilteredSignal.png}
\caption{Filtered Noisy Input Signal}
\end{figure}

\begin{figure}
\includegraphics[width=\linewidth]{CFFTFilteredSignal.png}
\caption{CFFT Filtered Signal}
\end{figure}

\begin{figure}
\includegraphics[width=\linewidth]{CFFTFilteredLowFreqVibrationNoise.png}
\caption{CFFT Filtered Low Frequency Vibration Noise}
\end{figure}

\begin{figure}
\includegraphics[width=\linewidth]{CFFTFilteredPosition.png}
\caption{CFFT Filtered Position Measurement Signal}
\end{figure}

\begin{figure}
\includegraphics[width=\linewidth]{CFFTFilteredHighFreqNoise.png}
\caption{CFFT Filtered High Frequency Noise}
\end{figure}


\newpage


\section{Error Analysis}

When working through this lab, I believe I was able to create adequate plots that met all of the required specifications. The hardest part for me was finding proper R, L, and C values. Our lab TA stated she chose a typical capacitor value and only changed her resistor and inductor values. That is the method I chose to work from. Once I chose an adequate value, I felt it easier for me to narrow in on the other values I needed to obtain. From this, I felt that I was able to obtain the rest of my plots that portrayed the requirements given in the lab handout. That being said, I am unsure if the plots are perfect. Based on the data I provided I believe they work, but I am sure there is a way to make them better. I think this lab challenging for that very reason, but I also think I hit the specifications that were required of us.



\section{Questions}

\textbf{1. Earlier this semester, you were asked what you personally wanted to get out of taking this course. Do you feel like that personal goal was met? Why or why not?} \\
{My initial goal for this lab was to gain a better understanding of what we were learning in ECE 350. I feel that this goal was met through the duration of this semester. I always found myself having a better understanding of the in-class topics after working on each of these labs. It has allowed to me excel in the class, in terms of homework and exam grades. I was also using this class/lab to see if I wanted to take ECE 450 in my next Fall Semester. This lab has made me want to do that. }




\section{Conclusion}

During this lab, I was challenged with trying to incorporate topics we had learned in previous labs with a few new topics. I though the lab handout provided useful example code that allowed me to incorporate the older labs and still meet the requirements given to us. I also thought the lab was successful in its goal of applying python concepts into a real world scenario. As I stated before, I felt that I was able to obtain the appropriate data that met all of the requirements given to us in the lab handout. This lab was challenging, but I think I learned a lot from it. I am excited to see where this takes me in the future. Here is the link to my github: \href{https://github.com/JMac1999}{Click Here}




\begin{thebibliography}{111}

  \bibitem{ACMT}
A. Aldroubi, C. Cabrelli, U. Molter, and Sui Tang,
Dynamical sampling, 
{\it  Applied and Computational Harmonic Analysis}, doi:10.1016/j.acha.2015.08.014, 2016


\end{thebibliography}
\end{document}

