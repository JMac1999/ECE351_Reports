%%%%%%%%%%%%%%%%%%%%%%%%%%%%%%%%%%%%%%%%%%%%%%%%%%%%%%%%%%%%%%%%
%                                                              %
% Jared McDaniel                                               %
% ECE 351 Section 51                                           %
% Lab 9                                                        %
% 11/2/2021                                                    %
% This file contains the contents of the Lab 8 Report          %
%                                                              %
%%%%%%%%%%%%%%%%%%%%%%%%%%%%%%%%%%%%%%%%%%%%%%%%%%%%%%%%%%%%%%%%


\documentclass[12pt]{report}
\usepackage[english]{babel}
%\usepackage{natbib}
\usepackage{url}
\usepackage[utf8x]{inputenc}
\usepackage{amsmath}
\usepackage{graphicx}
\graphicspath{{images/}}
\usepackage{parskip}
\usepackage{fancyhdr}
\usepackage{vmargin}
\usepackage{listings}
\usepackage{hyperref}
\usepackage{xcolor}

\definecolor{codegreen}{rgb}{0,0.6,0}
\definecolor{codegray}{rgb}{0.5,0.5,0.5}
\definecolor{codeblue}{rgb}{0,0,0.95}
\definecolor{backcolour}{rgb}{0.95,0.95,0.92}

\lstdefinestyle{mystyle}{
    backgroundcolor=\color{backcolour},   
    commentstyle=\color{codegreen},
    keywordstyle=\color{codeblue},
    numberstyle=\tiny\color{codegray},
    stringstyle=\color{codegreen},
    basicstyle=\ttfamily\footnotesize,
    breakatwhitespace=false,         
    breaklines=true,                 
    captionpos=b,                    
    keepspaces=true,                 
    numbers=left,                    
    numbersep=5pt,                  
    showspaces=false,                
    showstringspaces=false,
    showtabs=false,                  
    tabsize=2
}
 
\lstset{style=mystyle}

\setmarginsrb{3 cm}{2.5 cm}{3 cm}{2.5 cm}{1 cm}{1.5 cm}{1 cm}{1.5 cm}

\title{Lab 9}								
% Title
\author{ Jared McDaniel}						
% Author
\date{11/2/2021}
% Date

\makeatletter
\let\thetitle\@title
\let\theauthor\@author
\let\thedate\@date
\makeatother

\pagestyle{fancy}
\fancyhf{}
\rhead{\theauthor}
\lhead{\thetitle}
\cfoot{\thepage}
%%%%%%%%%%%%%%%%%%%%%%%%%%%%%%%%%%%%%%%%%%%%
\begin{document}

%%%%%%%%%%%%%%%%%%%%%%%%%%%%%%%%%%%%%%%%%%%%%%%%%%%%%%%%%%%%%%%%%%%%%%%%%%%%%%%%%%%%%%%%%

\begin{titlepage}
	\centering
    \vspace*{0.5 cm}
   % \includegraphics[scale = 0.075]{bsulogo.png}\\[1.0 cm]	% University Logo
\begin{center}    \textsc{\Large   ECE 351 - Section 51 }\\[2.0 cm]	\end{center}% University Name
	\textsc{\Large Fast Fourier Transform  }\\[0.5 cm]				% Course Code
	\rule{\linewidth}{0.2 mm} \\[0.4 cm]
	{ \huge \bfseries \thetitle}\\
	\rule{\linewidth}{0.2 mm} \\[1.5 cm]
	
	\begin{minipage}{0.4\textwidth}
		\begin{flushleft} \large
		%	\emph{Submitted To:}\\
		%	Name\\
          % Affiliation\\
           %contact info\\
			\end{flushleft}
			\end{minipage}~
			\begin{minipage}{0.4\textwidth}
            
			\begin{flushright} \large
			\emph{Submitted By :} \\
			Jared McDaniel  
		\end{flushright}
           
	\end{minipage}\\[2 cm]
	
%	\includegraphics[scale = 0.5]{PICMathLogo.png}
    
    
    
    
	
\end{titlepage}

%%%%%%%%%%%%%%%%%%%%%%%%%%%%%%%%%%%%%%%%%%%%%%%%%%%%%%%%%%%%%%%%%%%%%%%%%%%%%%%%%%%%%%%%%

\tableofcontents
\pagebreak

%%%%%%%%%%%%%%%%%%%%%%%%%%%%%%%%%%%%%%%%%%%%%%%%%%%%%%%%%%%%%%%%%%%%%%%%%%%%%%%%%%%%%%%%%
\renewcommand{\thesection}{\arabic{section}}
\section{Introduction}
 

The goal of this lab is to become familiar with fast Fourier transforms using Python.



   

\section{Equations}
{The equations below portray the ones that were given to us in the lab handout. Equation 1 is a cosine function, and it can be seen in Figures 1 and 4 under the Results section of this lab report. Equation 2 is a sine function, and it can be seen in Figures 2 and 5 under the Results section of this lab report. Equation 3 is a sine and cosine function, and it can be seen in Figures 3 and 6 under the Results section of this lab report. } 
 
 \begin{equation}
    \cos{(2\pi t)}
\end{equation}
\begin{equation}
    5\sin{(2\pi t)}
\end{equation}
\begin{equation}
    2\cos{((2\pi \cdot 2t)-2)}+\sin^{2}{((2\pi \cdot 6t)+3)}
\end{equation}



\section{Methodology}

This is my code for Part 1 Tasks 1, 2, and 3 of Lab 9. I started this section of code by defining my known values and my Fast Fourier Transform (fft). The fft was given to us in the lab handout, so it was easy to create. I then defined the three signals given to us in the lab handout, and plugged each of them into the fft. Then I imported the gridspec library to help me organize the plots each of my signals. After I got the gridspec command figured out, I utilized the plotting code from our previous labs to fully create each Figure. Figures 1, 2, and 3 correspond with this section of my code. 
\begin{lstlisting}[language=Python]
import numpy as np
import matplotlib.pyplot as plt
import scipy.signal as sig
import scipy.fftpack
import matplotlib.gridspec as gridspec

fs = 1e2
steps = 1/fs
t=np.arange(0, 2, steps)

#Part 1 Task 1, 2, and 3 Code

def FastFour(x, fs):

    N = len(x) # find the length of the signal
    X_fft = scipy.fftpack.fft(x) # perform the fast Fourier transform (fft)
    X_fft_shifted = scipy.fftpack.fftshift(X_fft) # shift zero frequency components
    # to the center of the spectrum
    freq = np.arange(-N/2 , N/2) * fs/N # compute the frequencies for the output
    # signal , (fs is the sampling frequency and
    # needs to be defined previously in your code
    X_mag = np.abs(X_fft_shifted)/N # compute the magnitudes of the signal
    X_phi = np.angle(X_fft_shifted) # compute the phases of the signal
    # ----- End of user defined function ----- #

    return freq, X_mag, X_phi

#equations given in the lab handout
E1 = np.cos(2*np.pi*t)
E2 = 5*np.sin(2*np.pi*t)
E3 = 2*np.cos((2*np.pi*2*t)-2)+np.sin((2*np.pi*6*t)+3)**2

freq1, X_mag1, X_phi1 = FastFour(E1, fs)
freq2, X_mag2, X_phi2 = FastFour(E2, fs)
freq3, X_mag3, X_phi3 = FastFour(E3, fs)

#plot 1 Signal 1
figure1 = plt.figure(figsize = (10, 7))
spec = gridspec.GridSpec(nrows = 3, ncols = 2, figure = figure1)
figure1.add_subplot(spec[0,:])
plt.ylabel ('x(t)')
plt.xlabel('t [s]')
plt.plot(t, E1)
plt.title("Magnitude of a Fourier Series Signal 1")
plt.grid()

figure1.add_subplot(spec[1,0])
plt.ylabel ('|x(f)|')
plt.stem(freq1, X_mag1, use_line_collection = True) #from lab handout
plt.grid()
figure1.add_subplot(spec[1,1])
plt.ylabel ('|x(f)|')
plt.stem(freq1, X_mag1, use_line_collection = True) #from lab handout
plt.xlim(-2,2)
plt.grid()

figure1.add_subplot(spec[2,0])
plt.ylabel ('<x(t)')
plt.stem(freq1, X_phi1, use_line_collection = True) #from lab handout
plt.xlabel('f [Hz]')
plt.grid()
figure1.add_subplot(spec[2,1])
plt.ylabel ('<x(t)')
plt.stem(freq1, X_phi1, use_line_collection = True) #from lab handout
plt.xlim(-2,2)
plt.xlabel('f [Hz]')
plt.grid()

\end{lstlisting}

{This is my code for Part 1 Task 4 of Lab 9. For this section of code we were tasked with creating a clean fft in order to clean up the plots in Figures 1, 2, and 3. To do this, I copied the fft from earlier in this lab and renamed it. Then I created a for loop that would cut off the Xphi values if Xmag became less than a certain value. Then I plotted the three signals again. Figures 4, 5, and 6 correspond with this section of my code.}
\begin{lstlisting}[language=Python]
def CleanFastFour(x, fs):

    N = len(x) # find the length of the signal
    X_fft = scipy.fftpack.fft(x) # perform the fast Fourier transform (fft)
    X_fft_shifted = scipy.fftpack.fftshift(X_fft) # shift zero frequency components
    # to the center of the spectrum
    freq = np.arange(-N/2 , N/2) * fs/N # compute the frequencies for the output
    # signal , (fs is the sampling frequency and
    # needs to be defined previously in your code
    X_mag = np.abs(X_fft_shifted)/N # compute the magnitudes of the signal
    X_phi = np.angle(X_fft_shifted) # compute the phases of the signal
    for i in range(len(X_phi)): 
        #this for loop should resolve the phase plots from Tasks 1, 2, and 3
        if (np.abs(X_mag[i]) < 1e-10): #from lab handout X_mag < 1e-10
            X_phi[i] = 0 #from lab handout X_phi = 0
    # ----- End of user defined function ----- #

    return X_mag, X_phi, freq

X_mag1, X_phi1, freq1 = CleanFastFour(E1, fs)
X_mag2, X_phi2, freq2 = CleanFastFour(E2, fs)
X_mag3, X_phi3, freq3 = CleanFastFour(E3, fs) 
\end{lstlisting}

{This is my code for Part 1 Task 5 of Lab 9. For this section of code we were tasked with running our Fourier Series Approximation from lab 8 through our clean fft. To do this, I copied my defined functions from lab 8, and placed it in my Clean fft function. I then created a plot using the plotting commands shown in the first part of my code. The plots from this task can be viewed in Figure 7 under the Results section of this lab report.    }
\begin{lstlisting}[language=Python]
t = np.arange(0, 16, steps) 
#plot starts at zero and finishes at 20 on the x-axis

def b(t, k):
    return (2/(k*np.pi))*(1-np.cos(k*np.pi))*np.sin(k*2*np.pi*t/8)

def FS(t, N):
    y = 0
    for k in range(1,N+1):
        y += b(t, k)
    return y

X_mag4, X_phi4, freq4 = CleanFastFour(FS(t,15), fs)
\end{lstlisting}

\newpage

\section{Results}

The results for this lab are shown below as plots. Each Figure holds 5 plots. The top plot is the original function, the middle plots show the fft of the magnitude (original and zoomed), and the bottom plots show the fft of the angle (original and zoomed). Figure 1 portrays the plots for Equation 1. Figure 2 portrays the plots for Equation 2. Figure 3 portrays the plots for Equation 3. Figures 4, 5, and 6 portray Equations 1, 2, and 3, but using the clean fft instead. Figure 7 portrays the plots of the Fourier series approximation we found in lab 8. 


\begin{figure}
\includegraphics[width=\linewidth]{Plot1.png}
\caption{Equation 1 Fourier Transform Plot}
\end{figure}

\begin{figure}
\includegraphics[width=\linewidth]{Plot2.png}
\caption{Equation 2 Fourier Transform Plot}
\end{figure}

\begin{figure}
\includegraphics[width=\linewidth]{Plot3.png}
\caption{Equation 3 Fourier Transform Plot}
\end{figure}

\begin{figure}
\includegraphics[width=\linewidth]{Plot1Clean.png}
\caption{Equation 1 Clean Fourier Transform Plot}
\end{figure}

\begin{figure}
\includegraphics[width=\linewidth]{Plot2Clean.png}
\caption{Equation 2 Clean Fourier Transform Plot}
\end{figure}

\begin{figure}
\includegraphics[width=\linewidth]{Plot3Clean.png}
\caption{Equation 3 Clean Fourier Transform Plot}
\end{figure}

\begin{figure}
\includegraphics[width=\linewidth]{Plot4Clean.png}
\caption{Fourier Series of the Square Wave}
\end{figure}
\newpage

\section{Error Analysis}

When working through this lab, I did experience difficulty with my first two figures. The bottom right plots were showing the wrong zero value. To solve this issue, our lab TA ran my code on her laptop and found that it worked correctly. She concluded that the Spyder library for .fft was updated. Other than that one issue, I thought the lab provided clear explanations that helped me complete it without any issues. 



\section{Questions}

\textbf{1. What happens if fs is lower? If it is higher? fs in your report must span a few orders of magnitude.}
{Fs is directly correlated to obtaining the sampling frequency. When it is lower, the output will be lower, and when it is higher, the output will be higher. }

\textbf{2. What difference does eliminating the small phase magnitudes make?}
{The difference that I see from eliminating the small phase magnitudes is it portrays the important data and leaves out the less important data. }

\textbf{3. Verify your results from Tasks 1 and 2 using the Fourier transforms of cosine and sine. Explain why your results are correct. You will need the transforms in terms of Hz, not rad/s.}
\begin{equation}
    FTT1 = \frac{1}{2}[d(f-1)+d(f+1)]
\end{equation}
\begin{equation}
    FFT1(0) = \frac{1}{2}[d(-1)+d(1)]
\end{equation}
\begin{equation}
    FFT1 = 0
\end{equation}
\begin{equation}
    FTT2 = \frac{5}{2}[d(f+1)-d(f-1)]
\end{equation}
\begin{equation}
    FFT2(0.25) = \frac{5}{2}[d(1.25)-d(-0.75)]
\end{equation}
\begin{equation}
    FFT2 = \frac{5}{2}
\end{equation}

{This corresponds to what I got in the lab.}

\textbf{4. Leave any feedback on the clarity of the expectations, instructions, and deliverables.}
{I think everything was clarified well in the Lab 9 document. I thought the example code was useful in creating the fft, and I thought the lab was easy to follow and complete.}



\section{Conclusion}

During this lab, I became more familiar with creating a Fourier Transform in python. I felt that the lab was successful in its goal of furthering our understanding of using Fourier Transforms, and it clarified several in-class topics for me. I don't have any recommendations to change this lab. I thought the lab provided all the necessary information for us to complete it. Here is the link to my github: \href{https://github.com/JMac1999}{Click Here}

\newpage


\begin{thebibliography}{111}

  \bibitem{ACMT}
A. Aldroubi, C. Cabrelli, U. Molter, and Sui Tang,
Dynamical sampling, 
{\it  Applied and Computational Harmonic Analysis}, doi:10.1016/j.acha.2015.08.014, 2016


\end{thebibliography}
\end{document}

