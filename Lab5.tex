%%%%%%%%%%%%%%%%%%%%%%%%%%%%%%%%%%%%%%%%%%%%%%%%%%%%%%%%%%%%%%%%
%                                                              %
% Jared McDaniel                                               %
% ECE 351 Section 51                                           %
% Lab 4                                                        %
% 9/28/2021                                                    %
% This file contains the contents of the Lab 4 Report          %
%                                                              %
%%%%%%%%%%%%%%%%%%%%%%%%%%%%%%%%%%%%%%%%%%%%%%%%%%%%%%%%%%%%%%%%


\documentclass[12pt]{report}
\usepackage[english]{babel}
%\usepackage{natbib}
\usepackage{url}
\usepackage[utf8x]{inputenc}
\usepackage{amsmath}
\usepackage{graphicx}
\graphicspath{{images/}}
\usepackage{parskip}
\usepackage{fancyhdr}
\usepackage{vmargin}
\usepackage{listings}
\usepackage{hyperref}
\usepackage{xcolor}

\definecolor{codegreen}{rgb}{0,0.6,0}
\definecolor{codegray}{rgb}{0.5,0.5,0.5}
\definecolor{codeblue}{rgb}{0,0,0.95}
\definecolor{backcolour}{rgb}{0.95,0.95,0.92}

\lstdefinestyle{mystyle}{
    backgroundcolor=\color{backcolour},   
    commentstyle=\color{codegreen},
    keywordstyle=\color{codeblue},
    numberstyle=\tiny\color{codegray},
    stringstyle=\color{codegreen},
    basicstyle=\ttfamily\footnotesize,
    breakatwhitespace=false,         
    breaklines=true,                 
    captionpos=b,                    
    keepspaces=true,                 
    numbers=left,                    
    numbersep=5pt,                  
    showspaces=false,                
    showstringspaces=false,
    showtabs=false,                  
    tabsize=2
}
 
\lstset{style=mystyle}

\setmarginsrb{3 cm}{2.5 cm}{3 cm}{2.5 cm}{1 cm}{1.5 cm}{1 cm}{1.5 cm}

\title{Lab 5}								
% Title
\author{ Jared McDaniel}						
% Author
\date{10/5/2021}
% Date

\makeatletter
\let\thetitle\@title
\let\theauthor\@author
\let\thedate\@date
\makeatother

\pagestyle{fancy}
\fancyhf{}
\rhead{\theauthor}
\lhead{\thetitle}
\cfoot{\thepage}
%%%%%%%%%%%%%%%%%%%%%%%%%%%%%%%%%%%%%%%%%%%%
\begin{document}

%%%%%%%%%%%%%%%%%%%%%%%%%%%%%%%%%%%%%%%%%%%%%%%%%%%%%%%%%%%%%%%%%%%%%%%%%%%%%%%%%%%%%%%%%

\begin{titlepage}
	\centering
    \vspace*{0.5 cm}
   % \includegraphics[scale = 0.075]{bsulogo.png}\\[1.0 cm]	% University Logo
\begin{center}    \textsc{\Large   ECE 351 - Section 51 }\\[2.0 cm]	\end{center}% University Name
	\textsc{\Large Step and Impulse Response of a RLC Band Pass Filter  }\\[0.5 cm]				% Course Code
	\rule{\linewidth}{0.2 mm} \\[0.4 cm]
	{ \huge \bfseries \thetitle}\\
	\rule{\linewidth}{0.2 mm} \\[1.5 cm]
	
	\begin{minipage}{0.4\textwidth}
		\begin{flushleft} \large
		%	\emph{Submitted To:}\\
		%	Name\\
          % Affiliation\\
           %contact info\\
			\end{flushleft}
			\end{minipage}~
			\begin{minipage}{0.4\textwidth}
            
			\begin{flushright} \large
			\emph{Submitted By :} \\
			Jared McDaniel  
		\end{flushright}
           
	\end{minipage}\\[2 cm]
	
%	\includegraphics[scale = 0.5]{PICMathLogo.png}
    
    
    
    
	
\end{titlepage}

%%%%%%%%%%%%%%%%%%%%%%%%%%%%%%%%%%%%%%%%%%%%%%%%%%%%%%%%%%%%%%%%%%%%%%%%%%%%%%%%%%%%%%%%%

\tableofcontents
\pagebreak

%%%%%%%%%%%%%%%%%%%%%%%%%%%%%%%%%%%%%%%%%%%%%%%%%%%%%%%%%%%%%%%%%%%%%%%%%%%%%%%%%%%%%%%%%
\renewcommand{\thesection}{\arabic{section}}
\section{Introduction}
 

The goal of this lab is to use Laplace transforms to find the time-domain response of an RLC band pass filter to impulse and step inputs.

\section{Equations}

The equations below portray the ones that were used during this lab. The first five equations were utilized to define the Sine method and used throughout the entire lab. The last equation provides the definition of the final value theorem for our calculated transfer function. This equation was portrayed in Part 2 of this lab, and its response can be viewed in Figure 3 from the Results section of this lab report.

\begin{equation}
    H(s) = \frac{\frac{s}{RC}}{s^{2}+\frac{s}{RC}+\frac{1}{LC}}
\end{equation}
\begin{equation}
    \alpha = \frac{-1}{2RC}
\end{equation}
\begin{equation}
   \omega = \frac{1}{2}\cdot {\sqrt{\frac{1}{RC}}^{2}}-4\cdot {\frac{1}{\sqrt{LC}}}^{2}
\end{equation}
\begin{equation}
   p = \alpha + \omega
\end{equation}
\begin{equation}
   Sine Method = \frac{|g|}{\omega}\cdot e^{\alpha t} \cdot \sin({\omega t + gangle})
\end{equation}
\begin{equation}
    \lim_{s\to 0} s(\frac{\frac{s}{RC}}{s^{2}+\frac{s}{RC}+\frac{1}{LC}}) = 0
\end{equation}


\section{Methodology}

This is my code for Part 1 of Lab 5. I started this section of code by copying my defined step function from Lab 4. Then I defined the Sine method from our textbook. To define the Sine method, I had to define all the variables and equations we discussed in class. Then I defined all the values from our circuit and plotted the function in python. The plot for the hand-solved time-domain impulse response can be viewed in Figure 1 under the Results section of this lab report.  
\begin{lstlisting}[language=Python]
#%% Part 1 Task 1 Code

import numpy as np
import matplotlib.pyplot as plt
import scipy.signal as sig

plt.rcParams.update({ 'font.size' :14}) #font size on plots

steps = 1.2e-5 #step size
t = np.arange(0, 1.2e-3 + steps, steps) 
#plot starts at zero and finishes at 0.0012 on the x-axis

def u(t): #step function
    y=np.zeros(t.shape) #y(t) is initialized as an array of zeros
    
    for i in range(len(t)):
        if(t[i]<0):
            y[i]=0 #step function is 0 when t < 0
        else:
            y[i]=1 #step function is 1 when t >= 0
    return y #return outputs stored in the array

def Sine(R,L,C,t): #defined sine method from our class text
    alpha = -1/(2*R*C) #alpha for our y(t) function
    omega = (1/2)*np.sqrt((1/(R*C))**2-4*(1/(np.sqrt(L*C)))**2 + 0.1j)
    #omega value for our y(t) function
    p = alpha + omega #defined p equation from our class text
    g = 1/(R*C)*p #g value for our y(t) function
    mag_g = np.abs(g) #magnitude of g
    phase = np.angle(g) #phase angle of g
    
    H = (mag_g/(np.abs(omega)))*np.exp(alpha*t)*np.sin(np.abs(omega)*t+phase)*u(t)
    return H

R = 1000
L = 27e-3
C = 100e-9
#defined values for R, L, and C from prelab 5

H = Sine(R, L, C, t) #setting up our impulse repsonse function
plt.figure(figsize = (10,7))
#Impulse Response plot
plt.plot(t, H)
plt.grid()
plt.ylabel('h(t)')
plt.xlabel('t')
plt.title('Part 1 Impulse Response')
\end{lstlisting}

{This is my code for Part 1 Task 2 of Lab 5. For this section of the lab, we were tasked with using the sig.impulse function to plot the s-domian transfer function from prelab. To do this, I first copied the example code from the Lab 5 handout. Then I copied the code I needed to plot a function from lab 4. The plot for this section of the lab can be viewed in Figure 2 under the Results section of this lab report. This figure will look exactly like Figure 1 from the Results section.  }
\begin{lstlisting}[language=Python]
#%% Part 1 Task 2 Code

num = [0 , 1/(R*C) , 0] # Creates a matrix for the numerator
den = [1 , 1/(R*C) , 1/(C*L)] # Creates a matrix for the denominator

tout, hout = sig.impulse((num,den), T = t)
#copied from lab 5 handout

plt.figure(figsize = (10,7)) 
#Impulse Response plot
plt.plot(tout, hout)
plt.grid()
plt.ylabel('h(t)')
plt.xlabel('t')
plt.title('Part 1 Impulse Response using Sig.Impulse')
\end{lstlisting}



{This is my code for Part 2 of Lab 5. In this part of the lab we were tasked with finding the step response of our H(s) function using the sig.step function. To do this, I copied the code from the Part 1 Task 2 of this lab and used sig.step instead of sig.impulse. Then I plotted the function. The plot can be viewed in Figure 3 under the Results section of this lab report. }
\begin{lstlisting}[language=Python]
#%% Part 2 Code

tout, hout = sig.step((num,den), T = t)
#copied from lab 5 handout,, but using sig.step instead of sig.impulse

plt.figure(figsize = (10,7)) 
#Impulse Response plot
plt.plot(tout, hout)
plt.grid()
plt.ylabel('H(s)*u(s)')
plt.xlabel('t')
plt.title('Step Response of H(s)')

#Part 2 Final value theorem 
#Using the final value theorem on the transfer function will be produce a 0
#the final value theorem takes the limit of s as it goes to 0.
\end{lstlisting}

\section{Results}

The results for this lab are shown below as plots. Figure 1 portrays the plot of the hand-solved time-domain impulse response from the prelab. Figure 2 portrays the impulse response when using the sig.impulse function. Figure 3 portrays the plots of final value theorem for the step response, H(s)u(s), using the sig.step function.    



\begin{figure}
\includegraphics[width=\linewidth]{Part 1 Task 1 Plot.png}
\caption{Hand Solved Time Domain Impulse Response}
\end{figure}

\begin{figure}
\includegraphics[width=\linewidth]{Part 1 Task 2 Plot.png}
\caption{S-Domain Impulse Response Using Sig.Impulse }
\end{figure}

\begin{figure}
\includegraphics[width=\linewidth]{Part 2 Plot.png}
\caption{Final Value Theorem for the Step Response}
\end{figure}



\newpage

\section{Error Analysis}

When working through this lab, I did not experience any difficulties. Once I understood that the Sine Method was best way to solve for the impulse response, the lab and prelab proved to be easy tasks to complete. 



\section{Questions}

\textbf{1. Explain the result of the Final Value Theorem from Part 2 Task 2 in terms of the physical circuit components.}
The final value theorem portrays our signal going to zero. The final value theorem can also be used with the transfer function to find the ratio of the output and input in the steady state after all transient components have decayed. Due to this, we can assume that our circuit's response only happens when it is transient and not steady state. 

\textbf{2. Leave any feedback on the clarity of the expectations, instructions, and deliverables.}
{I think everything was clarified well in the Lab 5 document. I did not struggle with any of the expectations or instructions given in this lab. I thought the lab was easy to follow and complete. I also thought that the example code given to us in the Lab 5 handout was useful in completing Part 2 of the the lab.  }



\section{Conclusion}

During this lab, I learned how to use Laplace transforms to compute the time domain response of an RLC bandpass filter into the impulse and step units. I felt that the lab was successful in its goal of furthering our understanding of Laplace transforms. I don't have any recommendations to change this lab. I thought the lab provided all the information necessary for us to complete it. Here is the link to my github: \href{https://github.com/JMac1999}{Click Here}

\newpage


\begin{thebibliography}{111}

  \bibitem{ACMT}
A. Aldroubi, C. Cabrelli, U. Molter, and Sui Tang,
Dynamical sampling, 
{\it  Applied and Computational Harmonic Analysis}, doi:10.1016/j.acha.2015.08.014, 2016


\end{thebibliography}
\end{document}

